\chapter{CHR in Prolog}

In dit hoofdstuk wordt een kort overzicht gegeven van de bestaande CHR implementaties in Prolog.

\section{Geschiedenis}

De eerste volledige CHR implementatie, zoals beschreven in \cite{christian:system}, werd gemaakt voor SICStus prolog (meer informatie in \cite{sicstus}). Deze werd geschreven door Christian Holzbaur en Thom Fr\"uhwirth, en wordt algemeen als de referentie-implementatie beschouwd. Een compatibel systeem werd later geschreven voor Yap \cite{yap}. Een oudere, onvolledige versie werd ook geschreven voor ECLiPSe \cite{eclipse}.

Een nieuwere implementatie was het K.U.Leuven CHR systeem \cite{tom:kulchr}. Het werd oorspronkelijk voor hProlog, maar bestaat ondertussen ook voor XSB Prolog \cite{xsb} en SWI-Prolog \cite{swiprolog}, zoals beschreven in \cite{tom:swi:wclp2005}).