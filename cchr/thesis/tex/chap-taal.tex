\chapter{De Taal}

In dit hoofdstuk wordt ingegaan op de taal ontwikkeld om CHR in C mogelijk te maken: CCHR.

Het doel van deze thesis was nagaan hoeveel effici\"enter dan bestaande CHR systemen een implementatie
in C kon zijn. De doelstellingen waar de taal aan moet voldoen zijn dan ook: \begin{itemize}
  \item We willen een taal ontwerpen die mogelijk maakt CHR en C te integreren: de CHR code moet C
        als host language kunnen gebruiken (C constructies als ``built-in constraints'' kunnen
	behandelen), en de C code moet bijvoorbeeld omgekeerd ook CHR constraints kunnen laten toevoegen.
  \item De taal moet het mogelijk maken heel efficiente code te genereren.
\end{itemize}

\section{Algemeen}

Om de integratie van C en CHR mogelijk te maken, zullen we een "cchr" blok toelaten binnenin een stuk normale
C code, wat vervangen zal worden door een cchr-compiler door een stuk equivalente pure C code. In die zin moet
CCHR dan ook eerder als een taal-uitbreiding van C beschouwd worden, en niet als aparte taal. Alle C taal-elementen
die de omgeving wijzigen in de ``omhullende'' C code zijn automatisch ook van toepassing op CHR (bv. \#include directives).

\section{Syntax}

Om de syntax in te leiden, is het interessant te beginnen met een eenvoudig voorbeeld (zie~\ref{fib:exCode}).

\begin{exCode}
\begin{Verbatim}[frame=single,numbers=left]
#include <stdio.h>
#include <stdlib.h>

#include "fib_cchr.h"

cchr {
  constraint fib(int,long long),init(int);

  begin @ init(_) ==> fib(0,1LL), fib(1,1LL);
  calc @  init(Max), fib(N2,M2) \ fib(N1,M1) <=>
    alt(N2==N1+1,N2-1==N1), N2<Max |
    fib(N2+1, M1+M2);
}

int main(int argc, char **argv) {
  cchr_runtime_init();
  cchr_add_init_1(90);
  cchr_consloop(j,fib_2,{
    printf("fib(%i,%lli)\n", 
      cchr_consarg(j,fib_2,1),
      (long long)cchr_consarg(j,fib_2,2));
  });
  cchr_runtime_free();
  return 0;
}
\end{Verbatim}
\caption{\label{fib:exCode} Fibonacci-voorbeeld}
\end{exCode}
