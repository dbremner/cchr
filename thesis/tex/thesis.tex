% Document class and font size report, article, book,..
\documentclass[dutch,12pt,oneside,a4paper]{book}

% Make sure you can process texts with accents
\usepackage[dutch]{babel}
\usepackage[latin1]{inputenc}
\usepackage{THcover}

% Handle differences between PdfLaTeX/LaTeX
\ifx\pdftexversion\undefined
\usepackage{url}
\usepackage[dvips]{geometry}
\usepackage[dvips]{graphics}
\newcommand{\phantomsection}{}
\newcommand{\includepdf}[1]{}
\newcommand{\texorpdfstring}[2]{#1}
% In LaTeX, include a converted .eps
\newcommand{\includebitmap}[2]{\includegraphics[width=#1\textwidth]{#2}}
\newcommand{\includevector}[2]{\includegraphics[width=#1\textwidth]{#2}}
\else
\usepackage[pdftex]{geometry}
\usepackage[pdftex]{graphics}
\usepackage{pdfpages}
\usepackage[
        pdftex,
        pdfpagelabels=true,
        plainpages=false,
        colorlinks=true,
        breaklinks=true,
        linktocpage=true,
        pdfstartview=FitV,
        linkcolor=black,
        citecolor=black,
        urlcolor=black
]{hyperref}
% In PdfLaTeX, include a .png directly
\newcommand{\includebitmap}[2]{\includegraphics[width=#1\textwidth]{#2.png}}
\newcommand{\includevector}[2]{\includegraphics[width=#1\textwidth]{#2.pdf}}
\fi

\geometry{%
        a4paper,
        top=2cm,
        left=2cm,
        textwidth=17cm,
        textheight=25cm,
}

% Use special math symbols
\usepackage{amsmath}

% Set paragraph layout
\setlength{\parindent}{0mm}
\setlength{\parskip}{3mm}

% Define symbols for natural, integer, real,.. numbers
\DeclareSymbolFont{AMSb}{U}{msb}{m}{n}
\DeclareMathSymbol{\N}{\mathbin}{AMSb}{"4E}
\DeclareMathSymbol{\Z}{\mathbin}{AMSb}{"5A}
\DeclareMathSymbol{\R}{\mathbin}{AMSb}{"52}
\DeclareMathSymbol{\Q}{\mathbin}{AMSb}{"51}
\DeclareMathSymbol{\I}{\mathbin}{AMSb}{"49}
\DeclareMathSymbol{\C}{\mathbin}{AMSb}{"43}

% Unit length for pictures
\setlength{\unitlength}{1cm}
\renewcommand{\thepage}{\arabic{page}.}

% fancy 'Verbatim' environnement
\usepackage{fancyvrb}

% multirows in tables
\usepackage{multirow}

% declare some extra floats
\usepackage{float}
\newfloat{exCode}{thp}{lop}[chapter]
\floatname{exCode}{Codevoorbeeld}

\usepackage{tabularx}

% load some common definitions
\newcommand{\code}[1]{{\tt #1}}


\begin{document}
% Example to generate the coverpages of a thesis with THcover.sty
% USE OF THIS STYLE, UNMODIFIED, IS MANDATORY !!
% Students experiencing problems must contact the TeX support team,
% preferably through their counsellor or supervisor,
% in particular for obtaining "special effects" (typically involving an other
% departement or university, a non-standard degree, etc...).
%
% In the title, student and supervisor names only capitalize proper names as in
% usual, written text; the style will convert things to (all) uppercase where
% needed.
% If some letter MUST remain lower case, use "{\lc[a-z]}".
% (for an example see the supervisor below)
%
\documentclass[12pt]{article}
\usepackage{THcover}
%
\usepackage[dutch]{babel}
% For theses in english delete the above line.

% Available degrees:
% CW	= burgerlijk ingenieur in de computerwetenschappen
% ACW	= aanvullende graad burgerlijk ingenieur in de computerwetenschappen
% LI	= licentie in de informatica
% AOTI  = aanvullende opleiding toegepaste informatica
% MS    = master of science (in informatics)
% MAI	= master of artificial intelligence
% MAIC	= master of artificial intelligence and cognitive science
% MECS	= master of engineering in computer science
% MEAI	= master of engineering in artificial intelligence
%
\degree{CW}

% Available options for CW students; this dates from the past
% TW    = richting toegepaste wiskunde
% PR    = richting programmatuur
% ME    = richting mechatronica
% For non-CW students this command is ignored.
% \cwoption{??}

% As it is not unusual for several students to cooperate on their thesis,
% on rare occasions, these may be aspiring distinct degrees.
% In this case, and only in this case, add the other degree
% (which must be different from the first).
%\otherdegree{??}

% The style, by default, inserts the current academic year, which should be
% correct if you're generating this anywhere from January to September.
% If not, explicitly override the default (note the double dash!)
%\acyear{20??--20??}

% Before making this cover page, talk with your supervisor about the
% DEFINITIVE title. What was given to you in the beginning of the year
% was only a proposal!
\title{CCHR: De snelste CHR implementatie}

% If your title is in dutch, you need a translation into english.
% If your title is in english, you need a translation into dutch.
% You are not allowed to make this translation yourself.
% YOUR SUPERVISOR HAS TO GIVE THIS TO YOU!
\transtitle{CCHR: The fastest CHR implementation}

% 
% Titles will be split up automatically if necessary, or under your control by
% putting "\\" anywhere between words.

% Classify your thesis according to the ACM and/or AMS classification
% acm: http://www.acm.org/class/   Use the most recent version (currently: 1998)
% ams: http://www.ams.org/msc/
% Give different numbers in separate commands
\acm{D.3.2}
\acm{D.3.3}
\acm{D.3.4}
% \ams{??}

% Give your own name(s):
\student{Pieter}{Wuille}
% If there are two or more of you, simply specify several "\student"s.

% In case their are two distinct degrees, use "\student" to name those
% (one or more) aspiring the degree specified with "\degree", and use
% "\otherstudent" to name those (again one or more) aspiring the degree
% specified with "\otherdegree"; in this order. 
%\otherstudent{First}{Last}

% Give the titles, names and optionally affiliation
% (if different from the dept. of computer science) 
% of supervisor(s) [promotor(en)] (at least one),
% reader(s) [assessor(en)] and counsellor(s) [begeleider(s)] (zero or more).
% Examples are given below.
% Note that if affiliation is not specified, you can leave out the [] portion;
% however you must put both sets of {}, even if the person has no titles.
% Note that most titles are always capitalized, but some never (see below).
\supervisor{Prof. Dr.}{B. Demoen}
\reader{Dr. ir.}{T. Schrijvers}
\reader{Prof. Dr.}{M. Denecker}
\counsellor{Dr. ir.}{T. Schrijvers}

\setupforcoverpage	% Leave these three lines in !!
\begin{document}
\begin{coverpage} 

% \begin{abstract}
%  Now this is some kind          % or it can be input-ed from file
%  of an abstract.                % e.g., \input{abstract}
% 
%  As always, you can divide your text over several paragraphs simply by leaving
%  a blanc line.
%  You can {\em emphasize\/} or {\bf embolden} some words if you feel like doing
%  so, or even put them in {\sc small Capitals},
%  just don't {\em ex\/}{\sc ag}ger{\bf ate}.
%  Accents are equally simple, e.g., ``{\'e}{\"\i}{\c{c}}{\`a}{\^o}''.
%  (Note the use of {\"\i} rather than {\"i} which is quite ugly.)
%  % Be careful with TeX's special symbols #$%&^_~{}\ !!
% 
%  {\bf Use of this style, unmodified, is mandatory !!}
% 
%  Students experiencing problems must contact the \TeX\ support team,
%  preferably through their counsellor or supervisor.
% 
%  The simplest is to copy the file THcover.tex   
%  and adapt for your own needs;
%  naming it something ending in .tex (i.e., drop the .skel).
%  
%  Formatting and printing is done most easily with the local script ``tt~(1L)''
%  which will establish a \TeX\ environment if you do not already have one.
%  You only need the commands ``f{\em ormat\/}'' and ``p{\em rint\/}'', and
%  possibly ``s{\em how\/}'', if you wish to preview under X.
% \end{abstract}

\begin{abstract}
Hier komt abstract. Momenteel zijn er slechts enkele stukken geschreven, en citaten, referenties en bibliografie moeten nog toegevoegd worden.
\end{abstract}

\end{coverpage}		% Leave these two lines in !!
\end{document}



\cleardoublepage
\section*{Dankwoord}

Hier komt mijn dankwoord. 

\newpage
\tableofcontents

\newpage
\mainmatter
\chapter{Inleiding}
\label{inleiding:chap}
\begin{quote}
{{\small\it Hier nog een citaat zetten.}}

{{\small\sc - Belangrijk individu}}
\end{quote}
\medskip

%Een goed boek over automatisch leren is \cite{mitchell:book}.

\part{CHR in C: CCHR}

\chapter{De Taal} \label{chap:taal}

In dit hoofdstuk wordt ingegaan op de taal ontwikkeld om CHR in C mogelijk te maken: CCHR.

Het doel van deze thesis was een effici\"ente CHR implementatie te schrijven die nauw kon interageren met C. Daarom is het noodzakelijk om: \begin{itemize}
  \item Een taal te ontwerpen die het mogelijk maakt CHR en C te integreren: de CHR code moet C
        als host-taal kunnen gebruiken (C constructies als ``built-in constraints'' kunnen
	behandelen), en de C code moet bijvoorbeeld omgekeerd ook CHR constraints kunnen laten toevoegen.
  \item De taal moet het mogelijk maken heel effici\"ente code te genereren.
\end{itemize}

\section{Algemeen} \label{sec:taal-gen}

Er wordt toegelaten een \code{cchr}-blok binnen normale C code te plaatsen, dat een beschrijving van CHR constraints en regels kan bevatten. Dit stuk zal later door de CCHR compiler omgezet naar C code zelf. In die zin moet CCHR dan ook eerder als een taal-uitbreiding van C beschouwd worden, en niet als aparte taal. 

\section{Syntax} \label{sec:taal-syn}

De algemene syntax bedacht om CHR met C te kunnen integreren, is sterk ge\"inspireerd door K.U.Leuven JCHR (zie \cite{peter:jchr}), waar CHR handlers geschreven worden binnen een specifiek blok in een bestand met \code{.jchr} extensie. In CCHR zal echter niet ge\"eist worden dat de CHR code zich in een apart bestand bevindt. In C is het veel minder de conventie om code te splitsen over verschillende bestanden dan in Java. Een \code{cchr}-blok stelt in C dan ook geen apart geheel voor, zoals een klasse, maar enkel een verzameling functies die op hetzelfde niveau komen te staan als de omliggende code. Hiermee is het ook mogelijk macro's te schrijven in de omliggende C code, die eveneens vanuit het cchr blok aan te roepen zijn.

Om de syntax in te leiden, is het interessant te beginnen met een eenvoudig voorbeeld (zie voorbeeld~\ref{fib:exCode}). Dit programma berekent opeenvolgende {\em getallen van Fibonacci}. Deze getallen zijn gedefinieerd als een rij die begint met twee maal een \'e\'en en waarbij alle volgende getallen de som zijn van hun 2 voorgangers.

\begin{exCode}
\begin{Verbatim}[frame=single,numbers=left]
#include <stdio.h>
#include <stdlib.h>

#include "fib_cchr.h" /* header gegenereerd door CCHR compiler */

cchr {
  constraint fib(int,long long),init(int);

  begin @ upto(_) ==> fib(0,1LL), fib(1,1LL);
  calc @  upto(Max), fib(N2,M2) \ fib(N1,M1) <=>
              alt(N2==N1+1,N2-1==N1), N2<Max |
              fib(N2+1, M1+M2);
}

int main(int argc, char **argv) {
  cchr_runtime_init();
  cchr_add_upto_1(90); /* voeg upto(90) toe */
  cchr_consloop(j,fib_2,{
    printf("fib(%i,%lli)\n", 
      cchr_consarg(j,fib_2,1),
      (long long)cchr_consarg(j,fib_2,2));
  });
  cchr_runtime_free();
  return 0;
}
\end{Verbatim}
\caption{\label{fib:exCode} Volledig fibonacci in CCHR --- \code{fib.cchr}}
\end{exCode}

\subsection{Het \code{cchr}-blok}

Een CCHR bronbestand bevat steeds een \code{cchr}-blok, ingeluid met het sleutelwoord \code{cchr} en gevolgd door een stuk code tussen accolades. Dit stuk code zal vervangen worden door een equivalent stuk C broncode. Er zal tevens een header gegenereerd worden met dezelfde bestandsnaam als het bronbestand, maar \code{.cchr} vervangen door \code{\_cchr.h}, en definities bevatten zodat met de CHR code ge\"interageerd kan worden. In voorbeeld~\ref{fib:exCode} kan u het \code{cchr}-blok zien op lijn 6 tot lijn 13. Lijn 4 bevat de opname van dat header bestand. Om problemen met recursieve definities op te lossen, is het nodig dit bestand op te nemen voor het \code{cchr}-blok.

Binnen het \code{cchr}-blok gelden dezelfde algemene regels als in C zelf: \begin{itemize}
  \item Commentaar wordt begonnen door \code{//} (tot op het einde van de lijn), of door \code{/*} (tot aan de eerstvolgende \code{*/}).
  \item Spaties en andere witruimte (nieuwe lijnen) hebben geen betekenis (tenzij als scheiding tussen 2 symboolnamen of operatoren).
\end{itemize}

\subsection{Constraints}

Constraints (zoals op lijn 7 van het voorbeeld), volgen JCHR's syntax: het \code{constraint} sleutelwoord gevolgd door een lijst van \'e\'en of meer constraint-namen, met tussen haakjes hun respectievelijke argumenten-types. Constraints zonder argument vereisen nog steeds \code{()} erachter, net zoals C een lege argumentenlijst vereist voor functies zonder argumenten (dit verschilt van JCHR).

Het is ook mogelijk om enkele opties aan te geven over constraints. Deze worden genoteerd door achter de argumenten van een constraint ``\code{option(}$optienaam$\code{,}$args$\ldots\code{)}'' te zetten (meerdere options mogelijk). Een lijstje van de
toegelaten opties: \begin{itemize}
  \item \code{fmt}: een standaard C printf formatstring, voor gebruik in debug mode (zie sectie~\ref{sec:debug}), om constraint suspensions te kunnen tonen. Na deze formatstring volgen de argumenten dat de formatstring zelf nodig heeft, waarbij naar de argumenten van de uit te printen constraints verwezen kan worden met \code{\$1}, \code{\$2}, \ldots.
  \item \code{init}: Een stuk C code dat uitgevoerd wordt bij het aanmaken van een constraint suspension van dit type. het kan een functie, een macro of gewoon een stuk code zelf zijn.
  \item \code{destr}: Een stuk C code dat uitgevoerd wordt bij het vernietigen van een constraint suspension van dit type. 
  \item \code{add}: Een stuk C code dat uitgevoetd bij bij het toevoegen van een constraint suspension van dit type in de constraint store. Hierbij kan naar de huidige constraint suspension verwezen worden met \code{\$0} (\code{\$0} is van het type \code{cchr\_id\_t}).
  \item \code{kill}: Een stuk C code dat uitgevoerd wordt bij het verwijderen van een constraint suspension van dit type uit de constraint store.
\end{itemize}

Het is niet mogelijk (in tegenstelling tot JCHR) om constraints met infix notatie te gebruiken. C zelf ondersteunt ook geen ``operator overloading'', dus deze functionaliteit leek ongepast.

\subsection{Symbolen}

Geldige namen voor constraints, functies, variabelen en andere C symbolen zijn letters (kleine en hoofdletters), cijfers en de underscore (\code{\_}). Het eerste teken mag geen cijfer zijn. Alle namen die met een hoofdletter beginnen kunnen dienen als CHR variabele. Dit zijn variabelen die gedefinieerd zijn door ze als argument van een constraint occurrence te gebruiken. Een naam (die nog niet eerder voorkwam) op een plaats gebruiken waar geen variabele gedefinieerd kan worden (zie verder) zal ervoor zorgen dat die als extern C symbool beschouwd wordt. Het is ook mogelijk een bepaalde naam sowieso als extern symbool te doen beschouwen, door het achter een ``\code{extern}'' sleutelwoord te zetten binnen het \code{cchr}-blok.

\subsection{regels} \label{sec:rules}

De syntax voor het noteren van CCHR regels is grotendeels gebaseerd op JCHR, waarvan de syntax sterk aanleunt bij de
originele CHR syntax. Ze bestaat uit \begin{enumerate}
  \item Een (optionele) benaming voor de regel, gevolgd door een \code{@}-symbool.
  \item Een of meerdere head-constraints, met argumenten (CHR variabelen of C expressies, zie verder), gescheiden door komma's.
  \item Eventueel een backslash (\code{$\backslash$}) gevolgd door nog \'e\'en of meer head-constraints (removed constraints, in geval van simpagation regel)
  \item Een regel-type aanduider (\code{==>} voor propagation of \code{<=>} voor simplification of simpagation).
  \item Eventueel een guard gevolgd door een vertikaal streepje (\code{|}).
  \item De body van de CHR regel.
  \item Afgesloten met een puntkomma (\code{;}).
\end{enumerate}

Er zijn enkele verschillen met JCHR: \begin{itemize}
  \item De regels staan niet in een apart \code{rules} blok. In JCHR wordt dit wel gedaan, maar dat lijkt een overbodige erfenis uit JaCK.
  \item Regels eindigen niet op een punt maar op een puntkomma. Een punt zou voor ambigu\"iteit zorgen, aangezien dat een geldige C operator is).
\end{itemize}

\subsubsection{Head}

De ``head'' constraints van een CHR regel (bestaande uit removed constraints en kept constraints) worden zoals vermeld genoteerd door met komma's gescheiden lijsten. Alle constraint-namen moeten in hetzelfde cchr-blok gedeclareerd zijn met het \code{constraint} sleutelwoord, en hun aantal argumenten (de ariteit) moet overeenkomen. Het is toegelaten meerdere constraints met dezelfde naam maar verschillende ariteit te hebben. Als argument kan een CHR variabele of een expressie gebruikt worden. Door een nog niet eerder gebruikte variabelenaam te schrijven wordt dit een CHR variabele. Elke andere uitdrukking wordt als expressie beschouwd. In een expressie is het niet mogelijk om een CHR variabele te definieren.

Variabelen die met een underscore (\code{\_}) beginnen, worden als anoniem beschouwd. Dat wilt zeggen dat ze niet meer verder gebruikt zullen worden. Dit gedrag komt overeen met Prolog's anonieme variabele (de underscore zelf), maar net zoals in JCHR en Prolog wordt toegestaan dat er nog andere letters volgen, wat leesbaarheid ten goede kan komen.

\subsubsection{Guard en Body}

Er is grote vrijheid aan wat als guard of body gebruikt mag worden in CCHR: \begin{enumerate}
  \item Een willekeurige C expressie, die tot 0 of niet-0 evalueert (false of true) \em{enkel guard}
  \item Een lokale variabele definitie. \em{zowel guard als body}
  \item Een stuk arbitraire C code (tussen accolades). \em{zowel guard als body}
  \item Een toe te voegen constraint (CHR of built-in). \em{enkel body}
\end{enumerate}

Binnen een guard wordt het sleutelwoord \code{alt} toegelaten. Dit geeft de mogelijkheid om twee of meer verschillende equivalente expressies te geven, om meer optimalisatie mogelijk te maken. In codevoorbeeld~\ref{fib:exCode} is hier een voorbeeld van te vinden op lijn 11. De betrekking als \code{N2==N1+1} en \code{N2-1==N1} schrijven maakt het de compiler mogelijk \code{N2} uit \code{N1} af te leiden maar ook \code{N1} uit \code{N2}. Zulke betrekkingen zouden in principe soms automatisch afgeleid kunnen worden, maar de CCHR compiler doet dit niet.

Lokale variabelen binnen een guard of body gebruiken kennen een eigen syntax, die verschilt van die van JCHR. Het volstaat om een datatype, gevolgd door een variabelenaam, een gelijkheidsteken, en eventueel een expressie voor initializatie te noteren. De naam van een dergelijke lokale variabele hoeft niet met een hoofdletter te beginnen. Een voorbeeldje: \begin{Verbatim}
  calc @ init(Max), fib(N,A) \ fib(N+1,B) <=> int sum=A+B, fib(N+2,sum);
\end{Verbatim}

\subsubsection{CHR Macro's}

Het is mogelijk om binnen het CCHR blok zelf verkorte notaties in te voeren voor gebruik binnen de body van CHR regels. Deze worden door de CHR compiler zelf verwerkt, wat verschilt van C macro's die door de C voorverwerker behandeld worden. Een voorbeeld is te vinden in codevoorbeeld~\ref{code:chrmacro}.
\begin{exCode}
\begin{Verbatim}[frame=single]
  chr_macro eqv(bigint_t,bigint_t) bigint_cmp($1,$2);
  chr_macro eqv(_,_) ($1==$2);
\end{Verbatim}
\caption{\code{chr\_macro} voorbeeld}
\label{code:chrmacro}
\end{exCode}
Dit zal het mogelijk maken om twee \code{bigint\_t} variabelen met elkaar te vergelijken met \code{eqv(a,b)}, maar ook twee andere variabelen met behulp van de C operator \code{==}. Het datatype \code{bigint\_t} zou elders gedefinieerd moeten zijn.

Zoals te zien is laat deze techniek toe dat verschillende data-types als parameters gebruikt worden (een soort polymorfisme), of dat \code{\_} als joker voor elk willekeurig type kan dienen. Indien er meerdere macrodefinities van toepassing zijn, wordt de eerste gebruikt. De datatypes zijn echter enkel bekend voor CHR variabelen, gedefini\"eerd in de head van een regel, of als lokale variabele. Expressies die geen loutere variabele zijn, kunnen enkel overeenkomen met het jokertype \code{\_}.

Het nut van deze macro's is een equivalent voorzien voor de {\em built-in constraints} van JCHR. In een latere uitbreiding zouden deze CHR macro's automatisch gegenereerd kunnen worden door het inladen van een extra module.

De macro \code{eq} is trouwens voorgedefinieerd om binaire equivalenties voor te stellen. Dit is een uitbreiding van de C operator \code{==} voor samengestelde datatypes (zoals structs).

\section{Variabelen} \label{sec:taal-var}

\subsection{Constante waarde}

In CCHR zijn CHR variabelen (of constraint argumenten) steeds onveranderlijk. Dat wilt zeggen dat de waarde die een argument krijgt bij de aanmaak van een constraint ongewijzigd zal blijven zolang die constraint suspension bestaat. Dit geldt niet voor lokale variabelen, wiens levensduur beperkt is tot de guard en body van een regel. De mogelijkheid bieden om argumenten aan te passen zou tot moeilijk te defini\"eren gedrag kunnen leiden.

Het constant zijn van argumenten wilt echter niet zeggen dat constraints niet gebruikt kunnen worden om wijzigbare data in op te slagen. Constante variabelen kunnen verwijzen naar een niet-constant gegeven in C. Een constante integer kan bijvoorbeeld een index zijn in een array die informatie bevat. In CHR wordt het wijzigen van argumenten typisch gebruikt om resultaten van bewerkingen terug te geven.
\begin{exCode}
\begin{Verbatim}[frame=single]
  calcMin1 @ min(N1,N2,R) <=> N1=<N2 | R=N1.
  calcMin2 @ min(_,N2,R) <=> R=N2.
\end{Verbatim}
\caption{Minimum in Prolog CHR}
\label{code:min-prologchr}
\end{exCode}
In codevoorbeeld~\ref{code:min-prologchr} zal het toevoegen van \code{min(A,B,C)} tot gevolg hebben dat R ge\"unificeerd wordt met de kleinste van A en B.

Dit implementeren in CCHR wordt bemoeilijkt door de afwezigheid van wijzigbare argumenten. Er zijn echter wel manieren om dit te vermijden.

\subsection{Pointers}

Een van de belangrijkste opzichten waarin C van recentere programmeertalen verschilt, is de mogelijkheid tot direct geheugenbeheer. Een C-programmeur kan zijn programma eender wat laten doen met het deel virtueel geheugen dat het (kan) krijgen van het systeem. Aanspreken van geheugen is mogelijk door gebruik te maken van zogenaamde {\em pointers}. Dit zijn variabelen die het geheugenadres van een andere variabele kunnen bevatten. Of het gebruik ervan de begrijpbaarheid van de erin geschreven programmas ten goede komt, wordt hier in het midden gelaten, maar het blijft een taalconstructie die voor veel mogelijkheden zorgt.

Pointers zijn \'e\'en mogelijkheid om een constraint argument met vaste waarde toch van ``betekenis'' te doen veranderen. Codevoorbeeld~\ref{code:min-cchr} geeft aan hoe het \code{min} programma in CCHR ge\"implementeerd zou kunnen worden met behulp van pointers.
\begin{exCode}
\begin{Verbatim}[frame=single]
  constraint min(int,int,int*); /* int* = pointer to int */
  calcMin1 @ min(N1,N2,R) <=> N1<=N2 | { *(R)=N1 };
  calcMin2 @ min(_,N2,R) <=> { *(R)=N2 };
\end{Verbatim}
\caption{Minimum in CCHR met pointers}
\label{code:min-cchr}
\end{exCode}

In dit voorbeeld is het laatste argument van \code{min} een pointer naar een \code{int} waarin het resultaat geplaatst wordt. De waarde van dat laatste argument blijft zolang de constraint bestaat hetzelfde, zijnde een verwijzing naar dezelfde geheugenplaats, maar de betekenis --- zijnde hetgeen op die bepaalde plaats staat --- wijzigt.

In dit voorbeeld wordt de pointer louter gebruikt wordt om een waarde terug te geven, vrij vergelijkbaar met het call-by-reference principe in imperatieve programmeertalen. Wanneer de betekenis van een dergelijke indirecte variabele echter als guard gebruikt zou worden, ontstaan er problemen.
\begin{exCode}
\begin{Verbatim}[frame=single]
  constraint facmult(int*,int*), mults();
  calcFac @ facmult(N,V) \ mults() <=> *(N)>0 
          | { *(V) *= *(N); *(N)--; }, mults();
\end{Verbatim}
\caption{Faculteiten in CCHR met pointers}
\label{code:fac-cchr}
\end{exCode}
Volgens de CHR {\em refined operational semantics} $\omega_r$ moet een constraint suspension waar een regel op van toepassing kan zijn die een guard heeft die waar geworden kan zijn, gereactiveerd worden. In codevoorbeeld~\ref{code:fac-cchr} wilt dat zeggen dat indien er een \code{facmult(N,V)} constraint suspension bestaat en een \code{mults()}, het verhogen van \code{*(N)} het reactiveren van de calcFac regel tot gevolg moet hebben. Hierbij zou \code{*(V)} met de oude waarde van \code{*(N)} vermenigvuldigd worden, en \code{*(N)} vervolgens met \'e\'en verlaagd. Dit blijft doorgaan tot \code{*(N)} gelijk is aan $0$, waarbij \code{*(V)} dus vermenigvuldigd is met de faculteit van de oorspronkelijke waarde van \code{*(N)}. De \code{mults()} constraint is nodig in dit artificieel voorbeeld zodat de \code{calcFac} regel blijvend uitgevoerd kan worden.

Het is echter moeilijk, zo niet onmogelijk, om effici\"ent te controleren wanneer de waarde van een expressie gewijzigd kan zijn, zeker in combinate met pointers die kunnen wijzen naar geheugenplaatsen die buiten de controle van het programma zelf kunnen wijzigen. Dit is zeker zo in combinatie met multi-threaded applicaties of wanneer gebruik gemaakt wordt van {\em Shared Memory} (SHM) technieken waarbij een deel virtueel geheugen gedeeld kan worden tussen verschillende programma's.

Daarom wordt de CCHR programmeur zelf verantwoordelijk gesteld voor aan te duiden wanneer een expressie die als guard gebruikt wordt gewijzigd kan zijn. De syntax hiervoor wordt in sectie~\ref{sec:crout-reactiv} aangereikt. Op zich is de programmeur niet verplicht voor deze reactivatie te zorgen, maar in dat geval verdwijnt de garantie dat het programma voldoet aan de verfijnde operationele semantiek $\omega_r$. Omwille van effici\"entie-redenen kan men toch opteren deze reactivate niet te doen, als men het gevolg ervan kent.

\subsection{Logische variabelen}

Om de mogelijkheden van CHR in C niet te beperken, is er ook ondersteuning voor echte logische variabelen. Ze worden echter algemeen voor C voorzien en niet enkel voor CCHR. Ze zorgen dat logische {\em built-in constraints} gebruikt kunnen worden in CCHR. Dit houdt in: \begin{itemize}
  \item Een waarde geven (een logische variabele heeft niet noodzakelijk een waarde).
  \item De waarde opvragen.
  \item Stellen dat een logische variabele gelijk is aan een andere logische variabele.
  \item Controleren of twee logische variabelen aan elkaar gelijk zijn.
\end{itemize}

Logische variabelen hebben ook de mogelijkheid om door de programmeur gespecifieerde routines aan te roepen bij bepaalde acties. Dit kan het bekendmaken dat reactivatie nodig kan zijn aanzienlijk vereenvoudigen. Op logische variabelen wordt teruggekomen in sectie~\ref{sec:logvar}.

\section{C routines}

Tot hiertoe werden enkel de mogelijkheden beschreven die CCHR code biedt. Het is echter ook noodzakelijk te specifi\"eren hoe C code kan interageren met de CCHR constraints. Er wordt ingegaan op de C routines die ter beschikking gesteld worden. Deze routines zullen in praktijk C functies of macro's zijn. Sommigen zijn algemeen voor een \code{cchr}-blok, en andere zijn specifiek voor bepaalde constraints.

\subsection{Initializatie en terminatie}

Vooraleer een CCHR constraint aan de {\em constraint store} mag toegevoegd worden, moet de store zelf ge\"initializeerd worden. Achteraf, wanneer geen gebruik van CCHR meer nodig blijkt, is het mogelijk alle geheugen geassocieerd met CCHR terug vrij te geven. Dit is inclusief de constraint store en alle constraint suspensions die erin opgeslagen zitten. Het gebeurt met de routines:
\begin{Verbatim}
  cchr_runtime_init();
  cchr_runtime_free();
\end{Verbatim}

\subsection{Toevoegen van constraints}

Vooraleer iets nuttig met CCHR gedaan kan worden, moet tenminste \'e\'en constraint aan de store toegevoegd worden. Om dit te doen wordt per constraint volgende functie voorzien: \begin{Verbatim}[commandchars=\\\{\}]
  void cchr_add_\argu{constraint}_\argu{ariteit}(\argu{arg1},\argu{arg2},\ldots);
\end{Verbatim}

\subsection{Reactivatie} \label{sec:crout-reactiv}

Soms is het nodig de CCHR runtime te informeren dat de waarde van een expressie in een guard gewijzigd zou kunnen zijn. Hiervoor worden volgende routines voorzien: \begin{Verbatim}[commandchars=\\\{\}]
/* alle constraint suspensions */
  cchr_reactivate_all(); 
/* alle constraint suspensions van bepaalde constraint */
  cchr_reactivate_all_\argu{constraint}_\argu{ariteit}();
/* enkel bepaalde constraint suspension */
  cchr_reactivate_\argu{constraint}_\argu{ariteit}(cchr_id_t {\em{PID}});
\end{Verbatim}

\subsection{Iteratie}

Uiteindelijk moet het mogelijk zijn de inhoud van de constraint store op te vragen. Hiervoor is volgende routine voorzien: \begin{Verbatim}[commandchars=\\\{\}]
  cchr_consloop(\argu{var},\argu{constraint}_\argu{ariteit},\argu{code})
\end{Verbatim}
\argu{code} is hierbij een stuk arbitraire C code dat voor elke constraint van type \argu{constraint} en ariteit \argu{ariteit} doorlopen wordt. Binnen \argu{code} kan de waarde van argumenten van de betrokken constraint met behulp van deze macro opgevraagd worden: \begin{Verbatim}[commandchars=\\\{\}]
  cchr_consarg(\argu{var},\argu{constraint}_\argu{ariteit},\argu{num})
\end{Verbatim}
waarbij \argu{num} naar het argument nummer $num$ verwijst (te beginnen tellen vanaf $1$). Een voorbeeldje is te vinden op lijnen 18 tot 22 van codevoorbeeld~\ref{fib:exCode}.

\chapter{Ontwerp en implementatie} \label{chap:impl}

In dit hoofdstuk gaan we in de op hoe ons CCHR systeem ontworpen en ge\"implementeerd is.

\section{Algemeen} \label{sec:impl-gen}

Zoals reeds aangehaald bestaat de implementatie van een CHR systeem normaal uit twee delen: de {\em compiler} en de {\em runtime}. De compiler zorgt voor een vertaling van de CHR-syntax naar code die uitvoerbaar is op het host-platform, en de runtime bevat alles wat noodzakelijk is om de vertaalde code te kunnen uitvoeren (algemene routines, onderhouden van de constraint store, \ldots).

Onze compiler is zelf in C geschreven en vertaalt CCHR code in enkele stappen tot normale C code, die dan verder door een standaard C compiler vertaald kan worden tot uitvoerbare code (machinetaal voor een specifiek platform). In tegenstelling tot Prolog en Java wordt het programma bij compilatie volledig tot machinetaal herleid, en is er dus geen {\em interpretatie} of {Just-in-time compilatie} meer nodig bij de uitvoering.

\section{De Compiler} \label{sec:impl-comp}

De compiler is het belangrijkste deel van het CCHR systeem. Het algemene concept is sterk gebaseerd op JCHR: we vertalen CHR broncode naar de host-language zelf, die dan door de bestaande compilers voor die taal verder gecompileerd kan worden tot een echt
uitvoerbaar programma. De compiler zelf begint met een parser en lexer om de taalstructuur van de CHR broncode te achterhalen, gevolgd door een omzetting naar een tussenvorm waarop enkele analyses gebeuren, en eindigt met een {\em template}-gebaseerde vertaling naar de uiteindelijke hosttaal. De precieze implementatie verschilt wel danig: \begin{itemize}
  \item De CCHR compiler vertaalt logischerwijs naar C en niet naar Java
  \item De CCHR compiler is zelf ook in C geschreven (de JCHR compiler was zelf ook in Java gemaakt)
  \item De gebruikte lexer en parser zijn gegenereerd met {\em Flex} en {\em Bison}, in plaats van {\em ANTLR}.
  \item In plaats van een extern pakket voor de templates te gebruiken, gebruiken we standaard C macro's.
 \end{itemize}
 
De grote fases zijn min of meer gescheiden van elkaar in de code. Ze zijn elk gedefinieerd in 1 of meerdere aparte bronbestanden, en de datastructuren die gebruikt worden voor de communicatie tussen de verschillende modules zijn op zich weer apart gedefinieerd. Zo zal de parser een {\em Abstract Syntax Tree} als resultaat geven, die enkel door de vertaal/analyse-module gebruikt wordt voor een omzetting naar een tussenvorm, waarop enkele statische analyses uitgevoerd kunnen worden, en die dan eenvoudig te gebruiken is door de codegeneratie om tot C macro's te vertalen.

\subsection{Algemeen}

De algemene werking van de CCHR compiler is als volgt: \begin{itemize}
  \item Alle op de commandolijn opgegeven bestanden worden doorlopen, en letterlijk gekopi\"eerd naar de uitvoer (C).
  \item Als in een van de bestanden een {\em cchr-blok} gevonden wordt: \begin{itemize}
    \item De {\em parser} wordt aangeroepen met dat cchr-blok als invoer.
    \item De {\em parser} roept zelf de {\em lexer} aan om syntactische elementen te herkennen.
    \item De {\em parser} bouwt een {\em abstract syntax tree} (AST).
    \item De AST wordt geanalyseerd, en een tussenvorm wordt opgebouwd.
    \item Op de tussenvorm worden optimalisaties doorgevoerd.
    \item Uiteindelijk wordt de tussenvorm opgezet naar een sequentie van C macro's.
    \item Deze C macro's worden in het uitvoerbestand (C) op de plaats gezet waar het {\em cchr-blok} stond.
  \end{itemize}
\end{itemize}

\subsection{De lexer}

De lexer is geschreven met behulp van Flex. Op de website van Flex vinden we: \begin{quote}
  Flex is a fast lexical analyser generator. It is a tool for generating programs that perform pattern-matching on text.
\end{quote}

Op basis van een bestand met definities van patronen, in de vorm van {\em regular expressions}, kan Flex een C bronbestand genereren dat heel snel een stuk input kan splitsen in de opgegeven patronen. 

Deze op deze wijze bekomen {\em lexer} vormt de eerste fase van het compilatie-proces. Ze herkent de opeenvolgende sleutelwoorden, operatoren, symbolen (kortweg {\em tokens}) van de brontaal, en geeft ze door aan de {\em parser}.

\subsection{De parser}

De parser is geschreven met behulp van Bison. Op de website van Bison vinden we: \begin{quote}
  Bison is a general-purpose parser generator that converts an annotated context-free grammar into an LALR(1) or GLR parser for that grammar.
\end{quote}

Er kan opgemerkt worden dat de werkwijze van Bison sterk lijkt op de CCHR compiler zelf. Er wordt ook uitgegaan van een andere taal die in C ingebed kan worden, en met behulp van een template-gebaseerde methode wordt pure C code gegenereerd.

Hiervoor is de grammaticale structuur van CCHR beschreven als een Bison {\em Context-Free Grammar}, met semantische acties erbij die een AST genereren. Er moet wel opgemerkt worden dat hoewel CCHR toelaat arbitraire C code op te nemen, de CCHR grammatica geen volledige C grammatica bevat. Ingebedde C code wordt namelijk niet volledig geparset, slechts tot op de hoogte dat noodzakelijk is
om het begin en het einde ervan te herkennen. Dat wil bijvoorbeeld zeggen dat \code{1+2*(3-4)} gewoonweg als \code{1 + 2 * ( 3 - 4 )} beschouwd wordt, en niet als \code{+(1,*(2,-(3,4)))}. Het letterlijk doorgeven van C expressies volstaat, aangezien alles toch nog
door de C compiler zelf moet.

Voor de {\em tokens} die de grammatica als basisblokken gebruikt, wordt beroep gedaan op de {\em lexer}.

Het resultaat hiervan is dus een AST, die echter helemaal niet geschikt is om converties en analyses op uit te voeren. Alle variabelen, constraints, \ldots zijn nog steeds beschreven als een hoop tekenreeksen. In de volgende stap wordt dit omgezet naar een werkbaar formaat. 

\subsection{De tussenvorm}

Na deze stap wordt de AST omgezet naar een nieuwe datastructuur, waarbij constraints, variabelen, regels, \ldots als aparte datastructuren in plaats van als tekenreeksen beschreven worden. Dit kan echter niet echt object-geori\"enteerd gebeuren, aangezien C dat niet ondersteunt.

De reden om de parser niet onmiddellijk via semantische acties deze tussenvorm te laten genereren is meer vrijheid in de taal te kunnen toelaten. Zo is het nu bijvoorbeeld mogelijk om een constraint pas te defini\"eren nadat die in een CHR regel gebruikt is.

Tijdens de omzetting van AST naar deze tussenvorm worden volgende transformaties doorgevoerd: \begin{itemize}
\item Alle verwijzingen naar constraints, variabelen, opties, \ldots worden herkend.
\item Alle regels worden omgezet naar HNF (Head Normal Form), waarbij alle expressies als parameters van contraints in de head die geen unieke variabelen zijn door een nieuwe variabele + een extra guard vervangen worden.
\item Macros worden vervangen door hun definitie.
\item Constraint occurrences worden bepaald (in welke rules en op welke plaats daarin elke constraint voorkomt).
\item Variabele occurrences worden bepaald (in welke constraint occurrence en op welke plaats daarin elke variabele voorkomt).
\item Afhankelijkheden tussen variabelen en statements worden bepaald.
\item Met deze afhankelijkheden wordt voor elke constraint occurrence een goede ``join ordering'' bepaald (zie sectie~\ref{ssec:joinorder}).
\end{itemize}

\subsection{Code generatie}

In de laatste fase van het vertalingsproces wordt de tussenvorm omgezet naar C code. Bij JCHR wordt van de template engine FreeMarker gebruik gemaakt. Het voordeel van templates gebruiken is duidelijk: de code die door de compiler zelf gegenereerd moet worden is algemener en beschrijft het proces op hoger niveau. Implementatie details zoals datastructuren kunnen dan onafhankelijk van de compiler uitgewerkt worden, wat het geheel flexibeler maakt en de kans op fouter beperkt.

In C bestaat echter reeds een gestandaardiseerd macro-systeem. We hebben er dan ook voor gekozen om deze C macros te gebruiken in plaats van een apart template engine. Het programma dat de macro-vertalingen doet, de C preprocessor, is standaard deel van het compilatieschema van C, waardoor het overbodig is om in de CCHR compiler deze vertaling te doen.

Het resultaat is dat het volledige template-vertaalproces verschoven wordt van de CCHR compiler naar het C compilatie-schema, en de uitvoer van de CCHR compiler is een sequentie van C macros in plaats van echte C code.

Het voordeel hiervan is dat de uitvoer van de CCHR compiler heel leesbaar blijft, en onafhankelijk blijft van enkele details. Zo is het mogelijk om een debug-versie van het CCHR programmatie te cre\"eren zonder de CCHR compiler opnieuw te moeten uitvoeren, enkel het resultaat ervan hercompileren met de C compiler en een andere optie volstaat. Het belangrijkste nadeel is de moeilijkheden dat het veroorzaakt bij het debuggen. 

\subsection{Uitvoer module}

Uiteindelijk roept de codegenerator een uitvoer module aan, die verantwoordelijk is voor de code mooi ge\"indenteerd weg te schrijven naar het uitvoerbestand, en ondertussen informatie bij te houden over het aantal geschreven regels. Dit is nodig omdat er lijnen van de vorm: \begin{Verbatim}
  #line "source.cchr" 16
  ...
  #line "source.c" 214
\end{Verbatim}
in de uitvoer gezet worden, die de C compiler hints geven over waar de code in het bestand vandaan kwam, om zinvollere waarschuwingen te kunnen geven. 

\section{Gegenereerde code} \label{sec:impl-code}

Zoals gezegd bestaat de gegenereerde code uit C macros. In dit stuk gaan we in op de structuur van die gegenereerde code.
Eerst geven we een inleiding op de C voorvertaler, en dan werken we een voorbeeld stap voor stap uit, om te eindigen bij de effectieve gegenereerde code die in de appendix te vinden is.

\subsection{C voorvertaler}

Eerst een korte inleiding over de C voorvertaler (``{\em preprocessor}'').

Het is een component die deel is van het standaard C compilatieproces (preprocessor, compiler, assembler, linker), die vooral gebruikt om platform-afhankelijke definites in te voegen in C programma. Zo bijvoorbeeld kan met het preprocessor {\em directive} \begin{Verbatim}
  #include <stdio.h>
\end{Verbatim}
Het {\em headerbestand} \code{stdio.h} ingeladen worden. Volgens de standaard zal dit bestand definities opnemen voor een aantal datatypes en functies nodig voor invoer/uitvoerroutines. 

Alle ``instructies'' die deze preprocessor kent heten {\em directives} (directieven), en moeten op een aparte lijn in het bronbestand staan, te beginnen met een hekje (\code{\#}). De belangrijkste directives die wij gebruiken zijn \code{\#include}, en \code{\#define}. Dat laatste dient om een macro te defini\"eren.

Macros zijn {\em tokens} die gedefinieerd worden als te substitueren door een reeks andere tokens. De eenvoudigste vorm, ook objectvorm genaamd, is: \begin{Verbatim}
  #define FOO bar(1);
\end{Verbatim}
wat aangeeft dat vanaf hier in de code ``\code{FOO}'' vervangen zal worden door ``\code{bar(1);}''. Macros kunnen echter ook parameters aannemen, de functionele vorm: \begin{Verbatim}
  #define FOO(par) bar(par1,par1+1);
\end{Verbatim}
waarbij bij voorbeeld de code ``\code{FOO(7)}'' vervangen zal worden door ``\code{bar(7,7+1);}''. Zulke macros hebben ook ondersteuning voor {\em variable arguments}: \begin{Verbatim}
  #define FOO(par,...) bar(par,__VA_ARGS__)
\end{Verbatim}
Hier zal ``\code{\_\_VA\_ARGS\_\_}'' de plaats innemen van alle argumenten die na par komen bij de vermelding van \code{FOO}. Zo zal bijvoorbeeld ``\code{FOO(sys,1,2)}'' vervangen worden door ``\code{bar(sys,1,2)}''.
De laatste mogelijkheid die we gebruiken is {\em token pasting}: \begin{Verbatim}
  #define FOO_1(arg) run(arg)
  #define FOO_2(arg) test(arg)
  #define BAR(par,sys) FOO_##par(sys)
\end{Verbatim}
De \code{\#\#} zorgt hier dat 2 tokens aan elkaar geplakt worden, en onderwerpt het resultaat terug aan macro-expansie. Zo zal in bovenstaand voorbeeld ``\code{BAR(1,2)}'' vervangen worden door ``\code{run(2)}'', maar ``\code{BAR(2,1)}'' door ``\code{test(1)}''.

In de komende tekst zal vaak verwezen worden naar macros, sommigen daarvan staan in het macro-definitie bestand, anderen worden door de compiler gegenereerd. Gegenereerde macros zullen we daarom ook steeds ``gegenereerde macros'' noemen, om verwarring te vermijden.

\subsection{Algemeen} \label{ssec:impl-code-alg}

Met het oog de gegenereerde macro-code niet te overladen met informatie die over heel het programma hetzelfde is, zoals de lijst van alle chr-constraints, is gekozen zoveel mogelijk op een algemene manier op te slagen. Dit vraagt een woordje uitleg.

Stel dat we dit CCHR blok zouden willen vertalen: \begin{Verbatim}
cchr {
  constraint fib(int,long long),init(int);

  begin @ init(_) ==> fib(0,1LL), fib(1,1LL);
  calc @  init(Max), fib(N2,M2) \ fib(N1,M1) <=>
    alt(N2==N1+1,N2-1==N1), N2<Max |
    fib(N2+1, M1+M2);
}
\end{Verbatim}
De volledige compiler uitvoer kan u vinden in sectie~\ref{sec:out-fib}.

De eerste belangrijke lijn die gegenereerd wordt is deze: 
\begin{Verbatim}
  #define CONSLIST(CB) CB##_D(fib_2) CB##_S CB##_D(init_1)
\end{Verbatim}
Deze lijn defini\"eert welke chr constraints allemaal bestaan. Ze is heel flexibel in gebruik, de aanroeper moet zelf 2 macros of functies voorzien: een voor het defini\"eren van een constraint, en een voor wat er tussen 2 constraints moet gebeuren. Zo is het mogelijk om code te laten genereren voor elke constraint, gescheiden met comma's, mits: \begin{Verbatim}
  #define CB_D(con) CODE_VOOR_CONSTRAINT(con)
  #define CB_S ,
  CONSLIST(CB)
\end{Verbatim}

Deze techniek wordt echter voor een stuk meer gebruikt dan enkel het aanduiden van de bestaande chr constraints. Er worden zulke index-macros gedefinieerd voor: \begin{itemize}
\item Welke chr constraints bestaan.
\item De occurrences van elke constraint.
\item Aantal kept/removed constraints in elke rule
\item Wat in propagation history bij te houden
\item Welke indexen nodig zijn, en waarover
\end{itemize}
Verder worden er nog gelijkaardige, maar eenvoudigere constructies gegenereerd voor constructor-, destructor-, add en kill routines per chr constraint.

Dan volgen nog enkele macros die de het mechanisme op hoog niveau beschrijven wat er voor elke constraint occurrence moet gebeuren. Hier gaan we zodadelijk op in.

Als afsluiter van de gegenereerde code staat een ``\code{CSM\_START}'', deze macro is gedefinieerd in het algemene macro-definitie bestand (zie runtime), en zal gebruikmakende van alle eerder gegenereerde macrodefinities expanderen tot de uiteindelijke C code. Daaruit volgt dat alle echte CCHR-code schijnbaar op de lijn van deze \code{CSM\_START} komt te staan.

\subsection{Constraint occurrences}

Voor elke constraint occurrence wordt een stuk code gegenereerd in de vorm van een aparte macro definitie. We zullen hier de basisopbouw geven, en later optimalisaties doorvoeren.

De naam die aan de gegenereerde macro voor een constraint occurrence moet van de vorm ``\code{CODE\_\argu{occurrence-naam}}'' zijn, en geen parameters hebben. De benaming voor de occurrence moet gewoon overal dezelfde zijn in de gegenereerde code. In praktijk gebruikt de compiler benamingen als ``\code{\argu{constraint}\_\argu{ariteit}\_\argu{rule}\_\argu{positie}}''. Positie is hierbij de letter \code{K} voor kept constraints, of de letter \code{R} voor removed constraints, gevolgd door een getal dat aanduidt de hoeveelste occurrence van dat type (removed of kept) het is binnen de gegeven rule, te beginnen bij 1.

Dan komen we bij de inhoud van deze macros. Het is hier dat het voordeel van een template-gebaseerde code generatie tot uiting komt: het algorithme wordt niet als C code gegenereerd, maar als een sequentie van macros. Deze set van macros is CSM gedoopt (Constraint Solver Macros), en een volledige lijst kan u vinden in appendix~\ref{chap:csm}. Het kan nuttig zijn de lijst erbij te nemen bij de komende uitleg, aangezien de precieze betekenis van de CSM macros hier niet meer uitgelegd wordt.

Als eerste versie vertrekken we van een imperatieve versie van de basisuitvoering beschreven in \cite{tomsphdthesis}:
\begin{itemize}
  \item Eerst ervoor zorgen dat de actieve constraint bestaat (\code{CSM\_MAKE}), en toegevoegd is aan de constraint store (\code{CSM\_NEEDSELF}).
  \item We itereren over alle partner constraints (de constraint occurrences in de huidige rule behalve de actieve), mbv. \code{CSM\_LOOP}.
  \item We controleren of er geen dubbels zijn in de verschillende partner constraints (mbv.  \code{CSM\_DIFF} en \code{CSM\_DIFFSELF} binnen een \code{CSM\_IF}).
  \item We controleren of de gevonden combinatie nog niet reeds geprobeerd is (mbv. \code{CSM\_CHECKHIST}).
  \item We defini\"eren alle lokale variabelen (met \code{CSM\_DECLOCAL} en \code{CSM\_DEFLOCAL}).
  \item We controleren of aan alle guards voldaan is (mbv. \code{CSM\_IF} en de constraint argumenten met \code{CSM\_ARG} en \code{CSM\_LARG} geschreven).
  \item We voegen de gevonden combinatie toe aan de propagation history (mbv. \code{CSM\_HISTADD}).
  \item We verwijderen eventuele removed constraints uit de constraint store (mbv. \code{CSM\_KILL} en \code{CSM\_KILLSELF}).
  \item We voeren de body van de chr rule uit, met de verwijzingen naar lokale variabelen vervangen door \code{CSM\_LOCAL}-macros en de constraint argument door \code{CSM\_ARG} en \code{CSM\_LARG}.
  \item Als de actieve constraint uit de constraint store verwijderd is, stop dan met de afhandeling (\code {CSM\_DEADSELF} en \code{CSM\_END}).
\end{itemize}

Met deze versie zijn enkele problemen, die misschien niet op het eerste zicht duidelijk zijn: \begin{itemize}
  \item Zodra een \code{CSM\_KILL} gebeurd is, kan \code{CSM\_LARG} niet meer gebruikt worden, omdat naar een onbestaande constraint verwezen kan worden (of erger nog: naar een andere constraint suspension die nu op die plaats staat). Daarom zullen we de constraint argumenten voor de uitvoering van de body opslagen in lokale (onwijzigbare) variabelen, mbv. \code{CSM\_IMMLOCAL}. Als er dan verder naar verwezen wordt in de body, gebruiken we gewoon \code{CSM\_LOCAL} ipv. \code{CSM\_LARG}.
  \item Het is mogelijk dat een constraint-argument verwijst naar een apart gealloceerd object, dat dmv. een destructor aangegeven is voor vernietiging bij verwijderen van de constraint suspension. Deze destructor kan echter niet aangeroepen worden door \code{CSM\_KILL}, omdat code verder in de body misschien nog naar het apart gealloceerd object kan verwijzen. Daarom voeren we een \code{CSM\_DESTRUCT} in voor elke \code{CSM\_KILL} of \code{CSM\_KILLSELF}, die pas na de body aangeroepen wordt.
  \item Wanneer de actieve constraint nog niet uit de constraint store verwijderd is, maar een van de voorgaande partner constraints al wel, dan moet onmiddellijk het volgende element van die iterator voor die partner constraint gekozen worden, om te vermijden dat de body toegepast zou worden voor een op dat moment niet meer bestaande constraint suspension. Daarom voeren we na de controle of de actieve constraint nog in de store zit, ook controles in voor alle \code{CSM\_LOOP}s, van buiten naar binnen, dmv. \code{CSM\_DEAD}, met een \code{CSM\_LOOPNEXT} op de betrokken variabele in. Deze controle is echter niet nodig voor de binnenste lus, aangezien daar sowieso onmiddellijk het volgende element gekozen zal worden.
\end{itemize}

\subsection{Optimalisaties}

Deze eerste versie van het compilatieschema is echter voor verbetering vatbaar. Voor een gedetailleerde uitleg over de optimalisaties en waarom ze toegestaan zijn, verwijzen we opnieuw naar \cite{tomsphdthesis}. 

{\bf Dubbels per type}: Het is enkel nodig om te controleren op dubbele constraint suspensions (\code{CSM\_DIFF} en \code{CSM\_DIFFSELF}) tussen suspensions van hetzelfde constraint type.

{\bf Propagation history}: Men hoeft enkel een propagation history bij te houden voor rules die geen removed constraints hebben. Deze laatste kunnen immers sowieso geen 2x uitgevoerd worden.

{\bf Destruction}: Indien een destructor aangeroepen moet worden voor een bepaalde constraint suspension, is het niet nodig hiermee te wachten tot na de uitvoering van de gehele body. Dit kan (meestal) gedaan worden zodra niet meer naar een variabele van die constraint verwezen wordt in de rule. De CCHR compiler zal de \code{CSM\_DESTRUCT} macro dan ook plaatsen na het laatste stuk body dat naar een variabele ervan verwijst.

{\bf Late Storage}: Het is wenselijk om het aanmaken van een constraint suspension, en zeker het eigenlijke toevoegen ervan aan de constraint store, zoveel mogelijk uit te stellen. Dit eerste bespaart geheugen, en dit tweede kan de snelheid ten goede komen, aangezien er ondertussen minder elementen in de store zijn om over te itereren. We zorgen ervoor dat de suspension aangemaakt is juist voor het zoeken naar partner constraints van een propagate-overgang\footnote{Een propagate-overgang komt in onze uitvoer overeen met een occurrence-macro waarbij de actieve constraint geen removed constraint is}, door de \code{CSM\_MAKE} enkel te plaatsen aan het begin van ``kept'' occurrence macros. Verder zorgen we er pas voor dat de constraint suspension in de store zit aan het begin van de uitvoering van de body van zo'n propagate-overgang. De \code{CSM\_NEEDSELF} wordt dus geplaatst voor de body van zo'n occurrence.

{\bf Simplification}: Bij simplificatie-overgangen is er zekerheid dat na uitvoering van de body, de actieve constraint zich niet meer in de store zal bevinden. We hebben dus geen \code{CSM\_DEADSELF} meer nodig om te controleren of een \code{CSM\_END} mag. Dit kan uitgebreid worden tot propagation-overgangen die partner constraints verwijderen. Hierbij kan de \code{CSM\_DEAD} conditie rond de \code{CSM\_LOOPNEXT} weggelaten worden. Daarbij komt nog dat na zo'n niet-conditionele \code{CSM\_END} of \code{CSM\_LOOPNEXT} geen verdere checks voor de diepere lussen meer nodig zijn, en dus volledig weggelaten kunnen worden. Het gebruik van deze expliciete \code{CSM\_LOOPNEXT} macros komt overeen met wat in JCHR ``backjumping'' genoemd wordt.

{\bf Generation}: Er kan aangetoond worden, dat indien tijdens de uitvoering van de body met een bepaalde actieve constraint suspension, diezelfde constraint suspension gereactiveerd werd, er geen nood meer is om nog verder te proberen rules erop toe te passen. Alle mogelijke rules zijn immers al toegepast tijdens de reactivatie. Dit wordt geimplementeerd door simpelweg in de definitie van \code{CSM\_DEADSELF} op te nemen, dat deze na reactivatie van zichzelf ook true is.

\subsection{Join ordering} \label{ssec:joinorder}

Een meer algemene verbetering die aangebracht kan worden aan voorgaand compilatieschema, is wat men ``join ordering'' noemt, zoals vermeld in \cite{duck:optimizing}. Tot hiertoe hebben we de volgorde waarin ge\"itereerd wordt over de verschillende partner constraints ongedefinieerd gelaten, maar een juiste keuze kan veel versnelling bij de uitvoering mogelijk maken.

In het algemeen komt het neer op zoveel mogelijk statements en voorwaardes die gecontroleerd worden voor we bij de uitvoering van de eigenlijke body zo snel mogelijk te doen, dwz. binnen zo weinig mogelijk lussen. Een overzicht: \begin{itemize}
  \item Guards (ook impliciete, door HNF convertie)
  \item Controles op dubbele constraints
  \item Propagation-history controles
  \item Definities van lokale variabelen
  \item Lokale statements in de guard, waar we tot hiertoe over zwegen.
\end{itemize}
In bovenstaand compilatieschema gebeuren al deze dingen pas zodra over alle partner constraints gelopen is, terwijl heel wat mogelijkheden al op voorhand uitgesloten zouden kunnen worden. Daarom gaan we proberen zoveel mogelijk van deze dingen reeds tussen de lussen door te controleren. Ze kunnen echter afhankelijk zijn van constraints, of van contraint-argumenten en lokale variabelen, die zelf ook van eerder gedefinieerde dingen afhankelijk kunnen zijn.

Hiervoor gebruiken we volgend algoritme in de compiler: \begin{itemize}
  \item We itereren over alle mogelijke volgordes van iteratie ($O(N!)$ combinaties, met $N$ het aantal partner constraints).
  \item Voor elke volgorde proberen we elke controle of statement zo vroeg mogelijk te plaatsen.
  \item We berekenen een ``score'', die aangeeft hoe snel deze volgorde verwacht wordt te zijn. Deze wordt bepaald door aan alle acties een gewicht toe te kennen, en deze te vermenigvuldigen met een factor voor elke lus waarbinnen ze staat.
  \item Uiteindelijk de volgorde met de laagste score te kiezen.
\end{itemize}
De gewichten die gebruikt worden zijn redelijk eenvoudig, en komen vermoedelijk niet overeen met de de realiteit. De bekomen volgorde zal daardoor vaak niet optimaal zijn, maar het verschil kan al wel aanzienlijk zijn.

\subsection{Indexen}

Na al deze verbeteringen blijft het meeste tijd verloren gaan in het opzoeken van de partner constraints. In sommige gevallen kunnen we echter al weten welke partner constraints nodig zijn, door indexen aan te leggen en te onderhouden die aangeven welke constraint suspensions allemaal een of meerdere waardes als bepaalde argumenten hebben. Hoe dit ge\"implementeerd is wordt uitgelegd in sectie~\ref{sec:impl-runtime}.

Om deze indexen te gebruiken in CSM, is het nodig ze eerst in een index-macro te defini\"eren zoals aangehaald in sectie~\ref{ssec:impl-code-alg}. Daarna is het ook nodig om in plaats van de traditionele \code{CSM\_LOOP} macro enkele andere macros te gebruiken. Eerst en vooral moet gedeclareerd worden dat een bepaalde variabele als index-iterator gebruikt gaat worden (met \code{CSM\_DEFIDXVAR}). Daarna moeten de op te zoeken waardes voor de verschillende argumenten ingevuld worden met \code{CSM\_SETIDXVAR}, en uiteindelijk moet ge\"itereerd worden met \code{CSM\_IDXLOOP} of \code{CSM\_IDXUNILOOP}. Over het verschil tussen beide gaan we dadelijk in.

Deze optimalisatie is degene die de belangrijkste snelheidsverbetering teweegbracht bij de implementatie, wat logisch is, bij sommige problemen maakt het de uitvoering een grootte-orde sneller door onmiddellijk de juiste constraint te vinden, in plaats van mogelijks er duizenden te moeten doorlopen.

Dit is ge\"implementeerd door bepaalde guards als speciaal te herkennen en aanleiding te laten geven tot indexen. Op dit moment gebeurt dit enkel voor logische (\code{==}) of binaire gelijkheden (\code{eq(\ldots)}, een eigen aanvulling). Hierbij komt ook het \code{alt} sleutelwoord kijken, dat zorgt voor verschillende mogelijkheden hoe de guard bekeken wordt. Uiteindelijk wordt tijdens de join ordering elke constraint waarvan 1 of meer argumenten door een simpele gelijkheid aan een expressie (van bekende waardes) bepaald kunnen worden, door een index-iterator beschreven en gezorgd, in plaats van door een normale iterator plus guard.

\subsection{Existenti\"ele en universele iteratoren}

Tijdens het uitvoeren van CCHR code wordt er ge\"itereerd over constraints in verschillende geneste lussen, en middenin die lussen worden er constraints verwijderd, toegevoegd, en gereactiveerd. Dit zorgt ervoor dat de toestand van de constraint store, en indien er indexen gebruikt worden, ook deze indexen tijdens het itereren op allerlei mogelijke manieren kunnen veranderen.

Iteratoren die bestand zijn tegen willekeurige wijzigingen en garanderen dat elk element dat uiteindelijk in de store zit ook effectief doorlopen is, zijn heel moeilijk effici\"ent te schrijven. We eisen dan ook niet dat \code{CSM\_LOOP} en consoorten eze functionaliteit aanbieden. Het is echter wel zo, dat constraint suspensions die tijdens het itereren toegevoegd worden aan de store, niet hoeven te worden gecontroleerd als mogelijke partner constraints, en dus niet noodzakelijk moeten overlopen worden door de iteratoren. Ze zijn immers reeds geactiveerd toen ze zelf toegevoegd werden, en ze zouden dus ondertussen reeds alle mogelijke transities met de huidige actieve constraint als partner constraint reeds moeten hebben doorgaan. Dit vergemakkelijkt de zaak duidelijk.

We zitten echter nog steeds met het probleem dat onze iteratoren wijzigingen in de constraint store moeten aankunnen tijdens het itereren. Blijkt echter dat dit niet altijd nodig is. Indien de simplification optimalisatie en expliciete backjumping ge\"implementeerd zijn, blijkt dat wanneer een rule minstens 1 removed constraint bevat (eventueel de actieve constraint), er na het toepassen van de body van een regel sowieso wordt gesprongen naar de volgende waarde voor de ``meest buitengelegen'' iterator voor een removed constraint, of verder. Hieruit volgt dat de lussen die daarbinnen zitten nooit moeten verderlopen eens de body is uitgevoerd.

Om die reden gaan we voor iteratoren waar het ingewikkeld is, 2 verschillende versies aanbieden: \begin{itemize}
\item een versie die slechts volgende mogelijkheden geeft totdat de constraint store gewijzigd kan zijn: de {\em existenti\"ele} iterator.
\item een versie die het algemene geval ook aankan: de {\em universele} iterator
\end{itemize}

Een universele iterator kan een veel grotere kost hebben om uitgevoerd te worden, wat ook in rekening wordt gebracht bij de kost berekening voor de join ordering.

\section{CSM Definities} \label{sec:impl-csm}

Bij SWI-Prolog en JCHR kon een duidelijk onderscheid gemaakt worden tussen gegenereerde code en runtime. Bij CCHR is het echter de vraag of we de definities van de CSM macros tot runtime kunnen rekenen. Het is namelijk geen code die tijdens de uitvoering nodig is, maar code die nodig is om de gegenereerde code naar een uitvoerbaar bestand te kunnen omzetten. Daarom zullen we de de daarin behandelde problemen hier in een aparte sectie bespreken.

De belangrijkste taak van CSM is de gebruikte datastructuren afschermen van de gegenereerde code, zodat deze beide onafhankelijk gewijzigd kunnen worden. We zullen daarom eerst even ingaan op de gebruikte datastructuren, en dan de implementatie ervan zelf bespreken.

\subsection{Noodzakelijke datastructuren} \label{ssec:impl-csm-ds}

{\bf Constraint store} De belangrijkste datastructuur is uiteraard degene voor de constraint store. Het moet een structuur zijn die toelaat heel snel constraints toe te voegen, te verwijderen en erover te itereren. De volgorde waarin is in principe niet zo van belang, maar het niet-teruggeven van elementen die tijdens het itereren zelf zijn toegevoegd is een pluspunt. Dynamische allocaties tijdens de uitvoering blijven liefst zo veel mogelijk beperkt.

{\bf Propagation history} Er moet ook een propagation history bijgehouden worden. Deze moet toelaten snel een bepaalde combinatie van constraint suspensions op te zoeken en verwijderen wanneer een van de betrokken constraint suspensions uit de store verwijderd wordt.

{\bf Indexen} Er is een datastructuur nodig om de indexen in bij te houden. Deze moet voor 1 of meerdere argumenten van een bepaalde constraint voor elke (combinatie van) waardes voor deze argumenten een lijst kunnen bijhouden van constraint suspensions die deze (combinatie van) waardes als argumenten heeft.
 
\subsection{De constraint store} \label{ssec:impl-csm-cs}

Voor de constraint store is gekozen voor een verzameling doubly-linked lists, die een gemeenschappelijk gealloceerd blok delen te gebruiken. Het gealloceerd blok wordt beschouwd als een array van elk elementen, waarbij elk element: \begin{itemize}
  \item Een verwijzing naar het volgende element heeft.
  \item Een verwijzing naar het vorige element heeft.
  \item Een uniek ID heeft (of $0$ voor ongebruikte elementen).
  \item Een datablok heeft.
\end{itemize}
De eerste elementen van de array worden niet echt gebruikt voor data in op te slagen, maar enkel als aanduidingen voor waar de lijsten voor elk type constraint beginnen. Aan elk contraint type wordt immers een getal van 0 tot N-1 toegekend.
De ``lege'' plaatsen in de array worden in een aparte (single) linked list bijgehouden, om snel een nieuwe plaats te kunnen innemen. 

\subsection{De propagation history} \label{ssec:impl-csm-ph}

\subsection{De index} \label{ssec:inpl-csm-index}

\section{Runtime} \label{sec:impl-rt}

Als ``runtime'' beschouwen we alle software-componenten die gemeenschappelijk zijn voor alle CCHR programma's, met uitzondering van de CSM definities die reeds behandeld zijn.

Er zijn slechts enkele stukken die nog overblijven om tot deze laag te beschouwen: \begin{itemize}
  \item De code voor de doubly-linked lists, die voor de constraint store gebruikt werd.
  \item De code voor de op union-find gebaseerde code om met logische variabelen te werken.
  \item De code om de hashtables te onderhouden die voor propagation history en indexen gebruikt werd.
  \item De code voor de hashfunctie voor bovenstaande hashtable.
\end{itemize}

\subsection{Doubly-linked lists} \label{sec:impl-rt-dll}

\subsection{Logische variabelen} \label{sec:impl-rt-log}

\subsection{De hashtable} \label{sec:impl-rt-ht}

\subsection{De hashfunctie} \label{sec:impl-rt-hf}

% Een van de meest algemene afwegingen die gemaakt moeten worden, is hoeveel de compiler doet, en hoeveel aan de runtime overgelaten wordt. In het algemeen stelt vast dat men hoe meer concepten door de compiler vertaald worden (en dus hoe eenvoudiger de runtime wordt), hoe effici\"enter het resultaat kan worden. De compiler is immers in staat
% 
% In het kader van effici\"entie, wat de belangrijkste doelstelling was, is de eigenlijke runtime beperkt. Enkel de code voor het
% bepalen van hashfuncties en de definitie van enkele fundamentele datatypes- en structuren kan echt als runtime beschouwd worden.
% Alle andere dingen worden (indirect) door de compiler gegenereerd. De eigenlijke output van de compiler is nog geen definitieve
% uitvoerbare C code, maar slechts C macro's die 





%
%\begin{figure}
%\framebox{\resizebox{\textwidth}{!}{\includegraphics{fig/bar}}}
%\caption{\label{bar:fig}An example of a bar graph.}
%\end{figure}



\cleardoublepage{}\phantomsection{}
\addcontentsline{toc}{chapter}{Appendices}

\appendix

\chapter{Programma-voorbeelden} \label{chap:output}

Hier worden enkele voorbeelden van compiler invoer en uitvoer gegeven.

\section{fib.cchr} \label{sec:out-fib}

\subsection{De cchr code}

\begin{Verbatim}[frame=single,numbers=left]
cchr {
  constraint fib(int,long long),init(int);

  begin @ init(_) ==> fib(0,1LL), fib(1,1LL);
  calc @  init(Max), fib(N2,M2) \ fib(N1,M1) <=>
    alt(N2==N1+1,N2-1==N1), N2<Max |
    fib(N2+1, M1+M2);
}
\end{Verbatim}

\subsection{De compiler output}

{\scriptsize
\begin{Verbatim}[frame=single,numbers=left]
#undef CONSLIST
#define CONSLIST(CB) CB##_D(fib_2) CB##_S CB##_D(init_1)

#undef PROPHIST_begin
#define PROPHIST_begin(CB,Pid1,...) CB##_I(Pid1,__VA_ARGS__,)
#undef PROPHIST_HOOK_begin
#define PROPHIST_HOOK_begin init_1
#undef RULE_KEPT_begin
#define RULE_KEPT_begin (1)
#undef RULE_REM_begin
#define RULE_REM_begin (0)
#undef ARGLIST_fib_2
#define ARGLIST_fib_2(CB,...) CB##_D(arg1,int,__VA_ARGS__) CB##_S CB##_D(arg2,long long,__VA_ARGS__)
#undef RULELIST_fib_2
#define RULELIST_fib_2(CB) CB##_D(fib_2_calc_R1) CB##_S CB##_D(fib_2_calc_K2)
#undef RELATEDLIST_fib_2
#define RELATEDLIST_fib_2(CB) CS##_D(init_1) CB##_S CS##_D(fib_2)
#undef FORMAT_fib_2
#define FORMAT_fib_2 "fib_2()"
#undef FORMATARGS_fib_2
#define FORMATARGS_fib_2 

#undef DESTRUCT_fib_2
#define DESTRUCT_fib_2(arg1,arg2) 
#undef CONSTRUCT_fib_2
#define CONSTRUCT_fib_2 
#undef ADD_fib_2
#define ADD_fib_2(PID) 
#undef KILL_fib_2
#define KILL_fib_2(PID) 
#undef RULEHOOKS_fib_2
#define RULEHOOKS_fib_2(CB,...) 

#undef CODELIST_fib_2_calc_R1
#define CODELIST_fib_2_calc_R1  \
  CSM_IMMLOCAL(int,N1,CSM_ARG(fib_2,arg1)) \
  CSM_IMMLOCAL(long long,M1,CSM_ARG(fib_2,arg2)) \
  CSM_DEFIDXVAR(fib_2,idx1,K2) \
  CSM_SETIDXVAR(fib_2,idx1,K2,arg1,CSM_LOCAL(N1)+1) \
  CSM_IDXLOOP(fib_2,idx1,K2, \
    CSM_IF(CSM_DIFFSELF(K2), \
      CSM_IMMLOCAL(int,N2,CSM_LARG(fib_2,K2,arg1)) \
      CSM_IMMLOCAL(long long,M2,CSM_LARG(fib_2,K2,arg2)) \
      CSM_LOOP(init_1,K1, \
        CSM_IMMLOCAL(int,Max,CSM_LARG(init_1,K1,arg1)) \
        CSM_IF(CSM_LOCAL(N2)<CSM_LOCAL(Max), \
          CSM_KILLSELF(fib_2) \
          CSM_ADD(fib_2,CSM_LOCAL(N2)+1,CSM_LOCAL(M1)+CSM_LOCAL(M2)) \
          CSM_DESTRUCT(fib_2,CSM_LOCAL(N1),CSM_LOCAL(M1)) \
          CSM_END \
        ) \
      ) \
    ) \
  ) \


#undef CODELIST_fib_2_calc_K2
#define CODELIST_fib_2_calc_K2  \
  CSM_MAKE(fib_2) \
  CSM_IMMLOCAL(int,N2,CSM_ARG(fib_2,arg1)) \
  CSM_IMMLOCAL(long long,M2,CSM_ARG(fib_2,arg2)) \
  CSM_DEFIDXVAR(fib_2,idx1,R1) \
  CSM_SETIDXVAR(fib_2,idx1,R1,arg1,CSM_LOCAL(N2) -1) \
  CSM_IDXUNILOOP(fib_2,idx1,R1, \
    CSM_IF(CSM_DIFFSELF(R1), \
      CSM_IMMLOCAL(int,N1,CSM_LARG(fib_2,R1,arg1)) \
      CSM_IMMLOCAL(long long,M1,CSM_LARG(fib_2,R1,arg2)) \
      CSM_LOOP(init_1,K1, \
        CSM_IMMLOCAL(int,Max,CSM_LARG(init_1,K1,arg1)) \
        CSM_IF(CSM_LOCAL(N2)<CSM_LOCAL(Max), \
          CSM_KILL(R1,fib_2) \
          CSM_NEEDSELF(fib_2) \
          CSM_ADD(fib_2,CSM_LOCAL(N2)+1,CSM_LOCAL(M1)+CSM_LOCAL(M2)) \
          CSM_DESTRUCT(fib_2,CSM_LOCAL(N1),CSM_LOCAL(M1)) \
          CSM_DEADSELF( \
            CSM_UNIEND(fib_2,R1) \
            CSM_END \
          ) \
          CSM_DEAD(R1, \
            CSM_LOOPNEXT(R1) \
          ) \
        ) \
      ) \
    ) \
  ) \



#undef ARGLIST_init_1
#define ARGLIST_init_1(CB,...) CB##_D(arg1,int,__VA_ARGS__)
#undef RULELIST_init_1
#define RULELIST_init_1(CB) CB##_D(init_1_begin_K1) CB##_S CB##_D(init_1_calc_K1)
#undef RELATEDLIST_init_1
#define RELATEDLIST_init_1(CB) CS##_D(fib_2)
#undef FORMAT_init_1
#define FORMAT_init_1 "init_1()"
#undef FORMATARGS_init_1
#define FORMATARGS_init_1 

#undef DESTRUCT_init_1
#define DESTRUCT_init_1(arg1) 
#undef CONSTRUCT_init_1
#define CONSTRUCT_init_1 
#undef ADD_init_1
#define ADD_init_1(PID) 
#undef KILL_init_1
#define KILL_init_1(PID) 
#undef RULEHOOKS_init_1
#define RULEHOOKS_init_1(CB,...) CB##_D(init_1,begin,__VA_ARGS__)

#undef CODELIST_init_1_begin_K1
#define CODELIST_init_1_begin_K1  \
  CSM_MAKE(init_1) \
  CSM_IMMLOCAL(int,_0,CSM_ARG(init_1,arg1)) \
  CSM_HISTCHECK(begin, \
    CSM_NEEDSELF(init_1) \
    CSM_HISTADD(begin,self_) \
    CSM_ADD(fib_2,0,1LL) \
    CSM_ADD(fib_2,1,1LL) \
    CSM_DEADSELF( \
      CSM_END \
    ) \
  ,self_) \


#undef CODELIST_init_1_calc_K1
#define CODELIST_init_1_calc_K1  \
  CSM_MAKE(init_1) \
  CSM_IMMLOCAL(int,Max,CSM_ARG(init_1,arg1)) \
  CSM_LOOP(fib_2,K2, \
    CSM_IMMLOCAL(int,N2,CSM_LARG(fib_2,K2,arg1)) \
    CSM_IMMLOCAL(long long,M2,CSM_LARG(fib_2,K2,arg2)) \
    CSM_IF(CSM_LOCAL(N2)<CSM_LOCAL(Max), \
      CSM_DEFIDXVAR(fib_2,idx1,R1) \
      CSM_SETIDXVAR(fib_2,idx1,R1,arg1,CSM_LOCAL(N2) -1) \
      CSM_IDXUNILOOP(fib_2,idx1,R1, \
        CSM_IF(CSM_DIFF(K2,R1), \
          CSM_IMMLOCAL(int,N1,CSM_LARG(fib_2,R1,arg1)) \
          CSM_IMMLOCAL(long long,M1,CSM_LARG(fib_2,R1,arg2)) \
          CSM_KILL(R1,fib_2) \
          CSM_NEEDSELF(init_1) \
          CSM_ADD(fib_2,CSM_LOCAL(N2)+1,CSM_LOCAL(M1)+CSM_LOCAL(M2)) \
          CSM_DESTRUCT(fib_2,CSM_LOCAL(N1),CSM_LOCAL(M1)) \
          CSM_DEADSELF( \
            CSM_UNIEND(fib_2,R1) \
            CSM_END \
          ) \
          CSM_DEAD(K2, \
            CSM_UNIEND(fib_2,R1) \
            CSM_LOOPNEXT(K2) \
          ) \
        ) \
      ) \
    ) \
  ) \



#undef HASHLIST_fib_2
#define HASHLIST_fib_2(CB,...) CB##_D(idx1,__VA_ARGS__) 
#undef HASHDEF_fib_2_idx1
#define HASHDEF_fib_2_idx1(CB,...) CB##_D(arg1,int,__VA_ARGS__) 

#undef HASHLIST_init_1
#define HASHLIST_init_1(CB,...) 

CSM_START
\end{Verbatim}
}

\section{Gebruik van logische variabelen}

In deze sectie worden enkele voorbeelden gegeven van de code die nodig is om met logische variabelen te werken.

\subsection{De Takeuchi functie} \label{sec:tak-cchr}

{\scriptsize
\begin{Verbatim}[frame=single,numbers=left]
/* first some empty definitions for logical code: we don't need reactivation */
#define log_int_cb_created(tag)
#define log_int_cb_merged(tag1,tag2)
#define log_int_cb_changed(tag)
#define log_int_cb_destrval(val)
#define log_int_cb_destrtag(tag)

/* define a log_int_t to be a logical uint64_t */
logical_header(int64_t,int,log_int_t)
logical_code(int64_t,int,log_int_t,log_int_cb)

/* the cchr block */
cchr {
  /* some macro's (overloaded!) to simplify usage of logical variables */
  macro set(log_int_t,log_int_t) log_int_t_seteq($1,$2);
  macro set(log_int_t,_) log_int_t_setval($1,$2);
  macro get(log_int_t) log_int_t_getval($1);
  macro copy(log_int_t) log_int_t_copy($1);
  macro new(log_int_t) log_int_t_create();
  macro del(log_int_t) log_int_t_destruct($1);

  constraint tak(int,int,int,log_int_t) option(destr,{del($4);}) option(init,{copy($4);})
  takid @ tak(X,Y,Z,A1) \ tak(X,Y,Z,A2) <=> { set(A1,A2); };
  
  taklow @ tak(X,Y,Z,A) ==> X <= Y | { set(A,Z); };
  takhi  @ tak(X,Y,Z,A) ==> X > Y | 
    log_int_t A1=new(A1), log_int_t A2=new(A2), log_int_t A3=new(A3), /* creation */
    tak(X-1,Y,Z,A1), tak(Y-1,Z,X,A2), tak(Z-1,X,Y,A3), 
    tak(get(A1),get(A2),get(A3),A),
    { del(A1); del(A2); del(A3); } /* destruction */
  ;
}
\end{Verbatim}
}

\subsection{Kleiner-dan of gelijk-aan} \label{sec:leq-cchr}

In het volgende voorbeeld wordt gebruik gemaakt van indexen in de logische variabele hun metadata om snel te itereren over de verschillende logische variabelen die een bepaalde waarde hebben, en voor reactivatie. Het is de bedoeling dat al de omslachtige code erbij om het te onderhouden, automatisch dmv. het ``logical'' sleutelwoord gegenereerd wordt.

{\scriptsize
\begin{Verbatim}[frame=single,numbers=left]
typedef struct {
  cchr_htdc_t leq_2_arg1;
  cchr_htdc_t leq_2_arg2;
  cchr_htdc_t leq_2_ra;
} log_int_t_tag_t;

logical_header(int,log_int_t_tag_t,log_int_t)

#define log_int_t_reactivate(var) { \
  CSM_LOGUNILOOP(leq_2,RA,leq_2_ra,log_int_t,var,{cchr_reactivate_leq_2(CSM_PID(RA));}) \
}

#define log_int_cb_created(val) { \
  cchr_htdc_t_init(&(log_int_t_getextrap(val)->leq_2_arg1)); \
  cchr_htdc_t_init(&(log_int_t_getextrap(val)->leq_2_arg2)); \
  cchr_htdc_t_init(&(log_int_t_getextrap(val)->leq_2_ra)); \
}
#define log_int_cb_merged(val1,val2) { \
  cchr_htdc_t_addall(&(log_int_t_getextrap(val1)->leq_2_arg1),&(log_int_t_getextrap(val2)->leq_2_arg1)); \
  cchr_htdc_t_addall(&(log_int_t_getextrap(val1)->leq_2_arg2),&(log_int_t_getextrap(val2)->leq_2_arg2)); \
  cchr_htdc_t_addall(&(log_int_t_getextrap(val1)->leq_2_ra),&(log_int_t_getextrap(val2)->leq_2_ra)); \
}
#define log_int_cb_changed(val) { \
  log_int_t_reactivate(val); \
}
#define log_int_cb_destrval(val)
#define log_int_cb_destrtag(val) { \
  cchr_htdc_t_free(&(log_int_t_getextrap(val)->leq_2_arg1)); \
  cchr_htdc_t_free(&(log_int_t_getextrap(val)->leq_2_arg2)); \
  cchr_htdc_t_free(&(log_int_t_getextrap(val)->leq_2_ra)); \
}


cchr {
  constraint leq(log_int_t,log_int_t) 
    option(init,{log_int_t_copy($1);log_int_t_copy($2);})
    option(destr,{log_int_t_destruct($1);log_int_t_destruct($2);}) 
    option(fmt,"leq""(#%i,#%i)""[%p,%p]" ,log_int_t_normalize($1)->_id,log_int_t_normalize($2)->_id ,$1,$2)
    option(add,{
      cchr_idxlist_t nw;nw.pid=CSM_PID($0); 
      nw.id=CSM_IDOFPID($0); 
      cchr_htdc_t_set(&(log_int_t_getextrap($1)->leq_2_arg1),&nw);
      nw.id=CSM_IDOFPID($0); 
      cchr_htdc_t_set(&(log_int_t_getextrap($2)->leq_2_arg2),&nw);
      nw.id=CSM_IDOFPID($0); 
      cchr_htdc_t_set(&(log_int_t_getextrap($1)->leq_2_ra),&nw);
      nw.id=CSM_IDOFPID($0); 
      cchr_htdc_t_set(&(log_int_t_getextrap($2)->leq_2_ra),&nw);
    })
    option(kill,{
      cchr_idxlist_t nw; 
      nw.pid=CSM_PID($0); 
      nw.id=CSM_IDOFPID($0); 
      cchr_htdc_t_unset(&(log_int_t_getextrap($1)->leq_2_arg1),&nw);
      cchr_htdc_t_unset(&(log_int_t_getextrap($2)->leq_2_arg2),&nw);
      cchr_htdc_t_unset(&(log_int_t_getextrap($1)->leq_2_ra),&nw);
      cchr_htdc_t_unset(&(log_int_t_getextrap($2)->leq_2_ra),&nw);
    })
    ;
  
  logical log_int_t log_int_cb;
  macro eq(log_int_t,log_int_t) log_int_t_testeq($1,$2);
  
  reflexivity @ leq(X,Y) <=> eq(X,Y) | true;
  antisymmetry @ leq(X1,Y1), leq(Y2,X2) <=> eq(X1,X2),eq(Y1,Y2) | {log_int_t_seteq(X1,Y1);};
  idempotence @ leq(X1,Y1) \ leq(X2,Y2) <=> eq(X1,X2),eq(Y1,Y2) | true;
  transitivity @ leq(X,Y1), leq(Y2,Z) ==> eq(Y1,Y2) | leq(X,Z);
}

logical_code(int,log_int_t_tag_t,log_int_t,log_int_cb)
\end{Verbatim}
}

\label{code:leq-c}

De C versie van het algoritme is als volgt:

{\scriptsize
\begin{Verbatim}[frame=single,numbers=left]
#define log_int_cb_created(val)
#define log_int_cb_merged(val1,val2)
#define log_int_cb_destrval(val)
#define log_int_cb_destrtag(tag)
#define log_int_cb_changed(val)

logical_header(int,int,log_int_t)
logical_code(int,int,log_int_t,log_int_cb)

/**
 * State a "LEQ" relation between 2 variables X and Y
 * size = number of variables
 * a = id of var X, b = id of var Y
 * vars = pointer to array of the variables. All variables
 *  have as value a number referring to their representative.
 *  When 2 variables are unified, one of them gets the
 *  representative of the other.
 * cmp = cmp[size*a+b] contains how often the implied
 *  "LEQ(X,Y)" constraint suspension was activated
 */
void addleq(int size,int a,int b,log_int_t *vars,int *cmp) {
  cmp[size*a+b]++; /* activate */
  a=log_int_t_getval(vars[a]);
  b=log_int_t_getval(vars[b]);
  if (a==b) { /* quit if already equal */
    return;
  }
  int ov=cmp[size*b+a]; /* remember "generation" of LEQ(Y,X) */
  if (ov) { /* LEQ(Y,X) exists, X and Y must be equal */
    log_int_t_seteq(vars[a],vars[b]); /* make them equal */
    int low=log_int_t_getval(vars[a]); /* get the representative 
      of the 2 (unified) variables */
    for (int j=0; j<size; j++) { /* make sure representative's generation
      is higher than both original's generation */
      cmp[j*size+low] = cmp[j*size+a]+cmp[j*size+b];
      cmp[low*size+j] = cmp[a*size+j]+cmp[b*size+a];
    }
    for (int j=0; j<size; j++) { /* re-activate existing LEQ's */
      if (cmp[j*size+low]) addleq(size,j,low,vars,cmp);
      if (cmp[low*size+j]) addleq(size,low,j,vars,cmp);
    }
    return;
  }
  for (int j=0; j<size; j++) {
    if (cmp[size*b+j]) { /* transitivity */
      addleq(size,a,j,vars,cmp);
      if (cmp[size*b+a]>ov) return; /* generation optimalisation */
    }
  }
}

void test(int size) {
  int *cmp=malloc(sizeof(int)*size*size);
  log_int_t *vars=malloc(sizeof(log_int_t)*size);
  for (int i=0; i<size; i++) {
    vars[i]=log_int_t_create();
    log_int_t_setval(vars[i],i);
  }
  memset(cmp,0,sizeof(int)*size*size);
  for (int i=0; i<size; i++) {
    addleq(size,i,(i+1)%size,vars,cmp);
  }
  for (int i=0; i<size; i++) {
    if (!log_int_t_testeq(vars[i],vars[(i+1)%size]))
      printf("outch %i != %i\n",i,(i+1)%size);
  }
  for (int i=0; i<size; i++) {
    log_int_t_destruct(vars[i]);
  }
  free(vars);
  free(cmp);
}
\end{Verbatim}
}

\section{De RAM simulator} \label{sec:ram-cchr}

{\scriptsize
\begin{Verbatim}[frame=single,numbers=left]
typedef enum { ADD,SUB,MULT,DIV,MOVE,I_MOVE,MOVE_I,CONST,JUMP,CJUMP,HALT } instr_t;

cchr {
	constraint mem(int, int) option(fmt,"mem(%i,%i)",$1,$2), 
	prog(int,int,instr_t,int,int) option(fmt,"prog(%i,%i,%i,%i,%i)",$1,$2,$3,$4,$5),
	prog_counter(int) option(fmt,"prog_counter(%i)",$1);
	
	extern ADD,SUB,MULT,DIV,MOVE,I_MOVE,MOVE_I,CONST,JUMP,CJUMP,HALT;
		
	constraint initmem(int) option(fmt,"initmem(%i)",$1);
	
	// add value of register B to register A
	iAdd @ prog(L,L1,ADD,B,A), mem(B,Y) \ mem(A,X), prog_counter(L) <=> mem(A,X+Y), prog_counter(L1);
	// subtract value of register B from register A
	iSub @ prog(L,L1,SUB,B,A), mem(B,Y) \ mem(A,X), prog_counter(L) <=> mem(A,X-Y), prog_counter(L1);
	// multiply register A with value of register B
	iMul @ prog(L,L1,MULT,B,A), mem(B,Y) \ mem(A,X), prog_counter(L) <=> mem(A,X*Y), prog_counter(L1);
	// divide register A by value of register B
	iDiv @ prog(L,L1,DIV,B,A), mem(B,Y) \ mem(A,X), prog_counter(L) <=> mem(A,X/Y), prog_counter(L1);

	// put the value in register B in register A
	iMove @ prog(L,L1,MOVE,B,A), mem(B,X) \ mem(A,_), prog_counter(L) <=> mem(A,X), prog_counter(L1);
	// put the value in register <value in register B> in register A
	iIMove @ prog(L,L1,I_MOVE,B,A), mem(B,C), mem(C,X) \ mem(A,_), prog_counter(L) <=> 
          mem(A,X), prog_counter(L1);
	// put the value in register B in register <value in register A>
	iMoveI @ prog(L,L1,MOVE_I,B,A), mem(B,X), mem(A,C) \ mem(C,_), prog_counter(L) <=> 
          mem(C,X), prog_counter(L1);

	// put the value B in register A        -> redundant if consts are in init mem
	iConst @ prog(L,L1,CONST,B,A) \ mem(A,_), prog_counter(L) <=> mem(A,B), prog_counter(L1);

	// unconditional jump to label A
	iJump @ prog(L,_L1,Instr,_,A) \ prog_counter(L) <=> Instr == JUMP | prog_counter(A);
	// jump to label A if register R is zero, otherwise continue
	iCjump1 @ prog(L,_L1,CJUMP,R,A), mem(R,X) \ prog_counter(L) <=> X == 0 | prog_counter(A);
	iCjump2 @ prog(L,L1,CJUMP,R,_A), mem(R,X) \ prog_counter(L) <=> X != 0 | prog_counter(L1);
	// halt
	iHalt @ prog(L,_L1,Instr,_,_) \ prog_counter(L) <=> Instr == HALT | true;

	// invalid instruction
	error @ prog_counter(_L) <=> {printf("eeeeik! error!!!\n");};
		
	init1 @ initmem(N) <=> N < 0  | true;
	init2 @ initmem(N) <=> N >= 0 | mem(N,0), initmem(N-1);
}
\end{Verbatim}
}

\chapter{CSM - De Constraint Solver Macros} \label{chap:csm}

In dit hoofdstuk wordt een overzicht gegeven van alle macro's die gedefinieerd zijn in CSM. 

\begin{table}
\begin{tabularx}{\textwidth}{|l|l|X|}
\hline
{\bf Macro} & {\bf argumenten} & {\bf Betekenis} \\
\hline
\code{CSM\_DIFF} & {\em var1}, {\em var2} & Controleer of {\em var1} en {\em var2} verschillende constraint suspensions zijn. \\
\code{CSM\_DIFFSELF} & {\em var} & Controleer of {\em var} een andere constraint suspension is dan de actieve constraint. \\
\code{CSM\_HISTCHECK} & {\em rule}, {\em code}, {\em susps\ldots} & Voer {\em code} enkel uit wanneer {\em rule} nog niet uitgevoerd is met {\em susps\ldots}. \\
\code{CSM\_IF} & {\em expr}, {\em code} & Voer {\em code} enkel uit wanneer aan {\em expr} voldaan is. \\
\code{CSM\_LOOP} & {\em constr}, {\em var}, {\em code} & Voer {\em code} uit voor alle constraint suspensions van type {\em constr} waarbij variabele {\em var} zal verwijzen naar de betrokken constraint suspension. \\
\code{CSM\_LOOPNEXT} & {\em var} & Ga onmiddellijk naar volgende element bij het itereren in {\em var}. \\
\hline
\end{tabularx}
\label{tab:csm-iter}
\caption{CSM macro's voor iteratie}
\end{table}

\begin{table}
\begin{tabularx}{\textwidth}{|l|l|X|}
\hline
{\bf Macro} & {\bf argumenten} & {\bf Betekenis} \\
\code{CSM\_ADD} & {\em constr}, {\em args\ldots} & Cre\"eer nieuwe constraint van type {\em constr}, met argumenten {\em args\ldots} \\
\code{CSM\_ADDE} & {\em constr} & Cre\"eer nieuwe constraint van type {\em constr}, zonder argumenten. \\
\code{CSM\_ARG} & {\em constr}, {\em naam} & Argument {\em naam} van de actieve constraint (type {\em constr}) opvragen \\
\code{CSM\_DEADSELF} & {\em code} & Voer {\em code} enkel uit als de actieve constraint ``dood'' is (niet in constraint store, of reeds gereactiveerd).\\
\code{CSM\_DEAD} & {\em var}, {\em code} & Voer {\em code} enkel uit als de niet-actieve constraint {\em var} ``dood'' is (niet in constraint store, of reeds gereactiveerd).\\
\code{CSM\_DECLOCAL} & {\em type}, {\em naam} & Definieer een (wijzigbare) lokale variabele \\
\code{CSM\_DEFLOCAL} & {\em type}, {\em naam}, {\em waarde} & Definieer en initializeer een (wijzigbare) lokale variabele \\
\code{CSM\_DESTRUCT} & {\em constr}, {\em args\ldots} & Roep de destructor aan voor constraint type {\em constr}, met argumenten {\em args\ldots}. \\
\code{CSM\_END} & & Be\"eindig afhandeling actieve constraint \\
\code{CSM\_HISTADD} & {\em rule}, {\em susps\ldots} & Voeg {\em susps\ldots} toe aan propagation history voor rule {\em rule}, zodat \code{CSM\_HISTCHECK} vanaf nu deze combinatie niet meer toelaat. \\
\code{CSM\_IMMLOCAL} & {\em type}, {\em naam}, {\em waarde} & Definieer en initializeer een (onwijzigbare) lokale variabele \\
\code{CSM\_KILL} & {\em var}, {\em constr} & Verwijder de niet-actieve constraint {\em var} (van type {\em constr}) uit de constraint store. Na deze aanroep mag {\em var} niet meer gebruikt worden, deze zou naar een onbestaande of zelfs andere constraint suspension kunnen verwijzen.\\
\code{CSM\_KILLSELF} & {\em constr} & Verwijder de actieve constraint (van type {\em constr} uit constraint store). \\
\code{CSM\_LARG} & {\em constr}, {\em var}, {\em naam} & Argument {\em naam} van niet-actieve constraint {\em var} (type {\em constr}) opvragen. \\
\code{CSM\_LOCAL} & {\em var} & Verwijs naar lokale variabele {\em var}). \\
\code{CSM\_MAKE} & {\em constr} & Maak actieve constraint suspension aan, van type {\em constr} (indien nog niet gebeurd) \\
\code{CSM\_NATIVE} & {\em code} & Een stuk C code uitvoeren \\
\code{CSM\_NEEDSELF} & {\em constr} & Zorg ervoor dat de huidige constraint (van type {\em constr}) in de constraint store zit (=erover ge\"itereerd kan worden). \\
\code{CSM\_START} & & Het genereren van de eigenlijke code \\
\hline
\end{tabularx}
\caption{Overige CSM macros}
\label{tab:csm-rest}
\end{table}

\begin{table}
\begin{tabularx}{\textwidth}{|l|l|X|}
\hline
{\bf Macro} & {\bf argumenten} & {\bf Betekenis} \\
\hline
\code{CSM\_DEFIDXVAR} & {\em constr}, {\em hash}, {\em var} & Geeft aan dat {\em var} gebruikt zal worden als iterator over hash {\em hash}. Wat in hash {\em hash} bewaard wordt, is met een aparte index-macro aangeduid. \\
\code{CSM\_SETIDXVAR} & {\em constr}, {\em hash}, {\em var}, {\em arg}, {\em val} & Geeft aan dat voor de index opzoeking in iterator {\em var} met hash {\em hash}, ge\"eist wordt dat argument {\em arg} van constraint {\em constr} gelijk is aan {\em arg}. \\
\code{CSM\_IDXLOOP} & {\em constr}, {\em hash}, {\em var}, {\em code} & Itereer in {\em var} over de gevraagde elementen, waarbij {\em code} elke keer uitgevoerd wordt. Deze iterator staat niet toe dat er voortge\"itereerd wordt nadat de constraint store aangepast is. \\
\code{CSM\_IDXUNILOOP} & {\em constr}, {\em hash}, {\em var}, {\em code} & Zelfde als \code{CSM\_IDXLOOP}, maar met ondersteuning voor wijzigen constraint store tijdens itereren. Merk wel op dat constraints die tijdens het itereren toegevoegd werden al dan niet voorkomen. Dit is normaal geen probleem, aangezien deze toch sowieso reeds geactiveerd werden. \\
\code{CSM\_LOGLOOP} & {\em constr}, {\em var}, {\em ent}, {\em type}, {\em arg}, {\em code} & Macro om over alle elementen van een bepaalde index, bijgehouden in de metadata van een logische variabele te itereren. {\em type} is het type logische variabele, {\em ent} is de te gebruiken index, {\em arg} is de betrokken logische variabele. \\
\code{CSM\_LOGUNILOOP} & {\em constr}, {\em var}, {\em ent}, {\em type}, {\em arg}, {\em code} & Analoog aan \code{CSM\_LOGLOOP}, maar met ondersteuning voor wijzigen constraint store tijdens itereren. \\
\code{CSM\_UNIEND} & {\em constr}, {\em var} & Nodig bij het voortijdig be\"eindigen van een ``uni'' lus (juist voor een \code{CSM\_END} of \code{CSM\_LOOPNEXT} van een meer naar buiten gelegen lus.\\
\hline
\end{tabularx}
\caption{CSM macros voor gebruik van indexen}
\label{tab:csm-idx}
\end{table}

\chapter{Gebruikte programma's} \label{chap:progs}

Hier wordt een overzicht gegeven van alle software die gebruikt werd bij het maken van deze thesis: \begin{itemize}
\item De benchmarks zijn uitgevoerd op een Gentoo Linux systeem, gebruik makende van een Linux 2.6.19 kernel.
\item De GCC (GNU Compiler Collection) versie 4.1.2 werd gebruikt om de CCHR compiler en de C en CCHR voorbeeldprogramma's te compileren.
\item Java 1.5.0.11 compiler en runtime en het K.U.Leuven JCHR systeem v1.5.1 werden gebruikt voor de voor de JCHR voorbeelden.
\item SWI-Prolog v5.6.17 werd gebruikt voor de Prolog-CHR voorbeelden.
\item De C, CCHR en Prolog voorbeelden gebruikten GMP (GNU Multiprecision Library) versie 4.2.1.
\item GNU Flex 2.5.33 werd gebruikt voor het genereren van de lexer.
\item GNU Bison 2.3 werd gebruikt voor het genereren van de parser.
\item Gnuplot 4.2rc4 werd gebruikt voor het maken van de grafieken.
\item Het schrijven van de code gebeurde met GNU Midnight Commander 4.6.1, Eclipse 3.1, en Kate 2.5.6.
\item Het typesetten van deze thesis gebeurde met de latex-distributie texlive 3.141592-1.30.5-2.2.
\end{itemize}



\backmatter
\addcontentsline{toc}{chapter}{Bibliografie}
\bibliography{thesis}
\bibliographystyle{alpha}

\addcontentsline{toc}{chapter}{Lijst van figuren}
\listoffigures
\addcontentsline{toc}{chapter}{Lijst van tabellen}
\listoftables

\end{document}
